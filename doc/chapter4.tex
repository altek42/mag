\chapter{Testy}
\label{rozdzial4}
Opis zakresu testów i jak będą przebiegać.
Każdy z poniższych testów będzie sprawdzany pod kątem poprawności wykonywania działań, czasu wykonywania w porównaniu do wykonania kodu na Node.js (dla bardziej złożonych testów z czasem będzie więcej), zajętość pamięci (również tylko przy tych których to ma sens) oraz porównaniu kodu z asemblerowego wygenerowanego za pośrednictwem języka C\#.

\section{Proste operacje matematyczne}
Test dodawania, odejmowania, mnożenia, dzielenia, przypisywania.
\section{Kolejność wykonywania działań}
Test na bardziej złożonych wyrażeniach. Sprawdzenie poprawności działania nawiasów oraz kolejności wykonywania działań.
\section{Wyrażenia warunkowe}
Test wyrażeń warunkowych. Kolejność wykonywania operacji and i or.
\section{Tablice}
Test obsługi tablic jedno i wielowymiarowych.
\section{Obiekty}
Test obsługi obiektów.
\section{Klasy}
Jeśli będzie implementacja.
Test działania obiektów klas.
\section{Funkcje}
Test działania funkcji.
1. Funkcja "void" bez parametrów.
2. Funkcja "void" z parametrami.
3. Funkcja zwracająca różne typy (proste, tablice, obiekty) bez parametrów.
4. Funkcje zwracająca różne typy z parametrami.
5. inne

\section{Algorytm 1}
\subsection{Opracowanie pseudokodu algorytmu 1}
\subsection{Implementacja algorytmu 1}
\subsection{Testy algorytmu 1}
\section{Algorytm 2}
\subsection{Opracowanie pseudokodu algorytmu 2}
\subsection{Implementacja algorytmu 2}
\subsection{Testy algorytmu 2}
