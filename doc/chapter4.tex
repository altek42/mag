\chapter{Testy}
\label{rozdzial4}

\par Podczas procesu tworzenia kompilatora, były również tworzone programy testujące dla poszczególnych funkcjonalności, w celu zweryfikowania prawidłowego działania programu. Dla każdego z modułów został utworzony program testujący w języku JavaScript, który obejmuje zakres funkcjonalności modułu. Zostały również utworzone tożsame programy w języku C\# w celu porównania do kodu języka JavaScript.
\par Zostały również wykorzystane dwa gotowe programy napisane w JavaScript odnalezione w Internecie. Również dla tych programów zostały utworzone odpowiedniki w języku C\# oraz zostały stworzone programy w języku JScript. Dla programów zostały przeprowadzone dodatkowe testy: został zmierzony czas wykonywania programu, zużycie pamięci oraz wielkości pliku wynikowego.

\section{Testy modułów}

\par A

\subsection{Porównanie wyniku wykonania programów}

\par Pierwszym z testów został wykonany program testujący wyświetlanie elementów na ekranie. Program wyświetla różne wartości różnych typów, takie jak wartości tekstowe w cudzysłowu podwójnym jak i pojedynczym, wartości liczbowe całkowite i rzeczywiste, wartości logiczne oraz tablice wartości. Na załączonym rysunku \ref{fig:result1} przedstawione są zrzuty ekranu wyniku działania programów w różnych wariantach.

\begin{figure}[!h]
	\centering
  \includegraphics[width=0.7\linewidth]{t1.png}
	\caption{Wyniki wykonania programu testowego dla modułu obsługi standardowego wyjścia. Źródło: własne}
	\label{fig:result1}
\end{figure}

\par Jak można zauważyć, wynik programu w języku C\# oraz JavaScript kompilowanych na wspólną platformę .NET są identyczne. Tutaj należy również wspomnieć, że standardowa biblioteka C\# nie posiada, żadnej z funkcji pozwalającej na wyświetlenie listy elementów tablicy przy pomocy jednej instrukcji. Tak jak podczas implementacji kompilatora JavaScript została zastosowana tutaj funkcja łącząca elementy tablicy w jeden ciąg znaków oraz zostały doklejone nawiasy otwierające oraz zamykające. 
\par Porównując wynik programu uruchomionego na platformie .NET z programem uruchomionym na platformie Node.js, można zauważyć nie wielkie różnice w wyświetlanych wynikach. Pierwszą z nich rzucającą się w oczy, jest zmiana koloru wyświetlanego tekstu dla liczb oraz wartości liczbowych. Jednak ten efekt zależny jest od formatowania kolorów w danej konsoli. Konsola wykryła wywołanie platformy Node.js, dzięki czemu zmieniła kolorystykę.
\par Drugim z różniącym się elementem jest sposób wyświetlania liczb rzeczywistych. Różnią się tym, że dla platformy .NET wyświetlany jest przecinek, kiedy na platformie Node.js wyświetlana jest kropka. Ostatnią różnicą w prezentowanych wynikach jest wyświetlanie wartości logicznych. Dla platformy .NET wartości są wyświetlane z wielką literą, a w przypadku Node.js wartości wyświetlane są małymi literami.

\par Kolejnym z testów dla którego wynik wykonania programu był różny, został przeprowadzony dla modułu działań arytmetycznych. Zrzuty wyników zostały przedstawione na rysunku \ref{fig:result2}. Głównie różnice występują przy wyświetlaniu wyniku operacji dzielenia.

\begin{figure}[!h]
	\centering
  \includegraphics[width=0.7\linewidth]{t2.png}
	\caption{Wyniki wykonania programu testowego dla modułu obsługi działania arytmetyczne. Źródło: własne}
	\label{fig:result2}
\end{figure}

\par Pierwszą z różnic między programem napisanym w języku C\# a JavaScript na platformie .NET jest wynik dzielenia dwóch liczb całkowitych. Wykonywane działanie ma postać $y = 10 / 15$, więc wynikiem jest liczba rzeczywista. W programie w języku C\# jest językiem silnie typowanym, co oznacza, że jeżeli przynajmniej jedna z liczb nie zostanie przekonwertowana na typ liczb rzeczywistych, to wynik będzie typu liczby całkowitej. W tworzonym kompilatorze została zaimplementowana automatyczna konwersja typów, w momencie kiedy zostanie wykryta operacja dzielenia dwóch liczb.
\par Drugą z widocznych różnic jest precyzja wartości zmiennych. W tworzonym kompilatorze JavaScript został wykorzystany typ pojedynczej precyzji, a w przypadku kompilatora C\# została zastosowana typ podwójnej precyzji. Porównując wyniki platformy .NET do platformy Node.js można zauważyć, że wartości na platformie .NET są zaokrąglane.

\par Przy uruchomieniu programu JScript jednego z algorytmów testujących, ukazała się jeszcze jedna różnica w prezentowanych na konsoli wynikach. Różnicą jest sposób wyświetlania tablicy elementów. W tworzonym kompilatorze, jak i na platformie .NET, elementy są otoczone nawiasem kwadratowym, oraz oddzielone są poza przecinkiem dodatkową spacją. W przypadku wyniku programu JScript, prezentowana tablica nie posiada nawiasów kwadratowych oraz liczby są oddzielone przecinkami bez spacji. Wynik programów zaprezentowany jest na rysunki \ref{fig:result3}

\begin{figure}[!h]
	\centering
  \includegraphics[width=0.7\linewidth]{t3.png}
	\caption{Wyniki wykonania programu dla algorytmu testowego nr 1. Źródło: własne}
	\label{fig:result3}
\end{figure}

\par Skrypty testowe dla pozostałych modułów nie wykazują różnic w prezentowanych wynikach na konsoli.

\subsection{Porównanie generowanego kodu assemblera}

\par Generowany kod assemblera z języka JavaScript będzie porównany z kodem dezasemblowanym kodem programu napisanego w języku C\#. Dezasemblacja jest wykonywana przy pomocy programu \textit{ildasm.exe} znajdujący się w pakiecie \textbf{.NET Framework}.
\par Pierwszą z widocznych różnic jest ilość deklaracji metadanych w nagłówku pliku. Kolejną z różnic jest ilość modyfikatorów przy deklaracji klasy jak i metody \texttt{Main}. Dla deklaracji klasy programu C\# zostały użyte następujące słowa kluczowe \texttt{private}, \texttt{auto}, \texttt{ansi}, \texttt{beforefieldinit}, a dla funkcji \texttt{Main} zostały użyte: \texttt{private} oraz \texttt{hidebysig}. 

\begin{figure}[!h]
	\centering
  \includegraphics[width=0.9\linewidth]{t4.png}
	\caption{Porównanie kodu asemblera dla deklaracji klasy oraz funkcji \texttt{Main}. Źródło: własne}
	\label{fig:result4}
\end{figure}

\par Kolejnym z różnic jest ilość deklarowanych zmiennych, która spowodowana jest słabą optymalizacją w tworzonym kompilatorze. Następną rzeczą jest nadawanie etykiet dla każdej z instrukcji w obrębie deklarowanych funkcji. 

\par Analizując generowane instrukcje kodu asemblera z dezasemblowanym kodem programu C\# można zauważyć różnicę przy zapisywaniu oraz odczytywaniu wartości zmiennych na stosie. W utworzonym kompilatorze odbywa się to zawsze poprzez nazwę zmiennej, a w .NET Framework wykorzystywane są instrukcje wykorzystujące indeksowanie zmiennych od wartości 0 do 3. Dodatkowo odwołanie się do kolejnych zmiennych wykorzystywana jest instrukcja w formie skróconej.


\begin{lstlisting}[language=IL, caption={Fragment kodu deasemblerowanego testu programu C\#, przedstawiający ładowanie wartości zmiennych na stos}, label=alg:asm]
  ...
  IL_0047:  ldloc.0
  IL_0048:  call       void [mscorlib]System.Console::WriteLine(string)
  IL_004d:  ldloc.1
  IL_004e:  call       void [mscorlib]System.Console::WriteLine(string)
  IL_0053:  ldloc.2
  IL_0054:  call       void [mscorlib]System.Console::WriteLine(int32)
  IL_0059:  ldloc.3
  IL_005a:  call       void [mscorlib]System.Console::WriteLine(float32)
  IL_005f:  ldloc.s    V_4
  IL_0061:  call       void [mscorlib]System.Console::WriteLine(bool)
  ...
\end{lstlisting}

\par Następne różnice generowanego kodu widoczne są przy operacjach arytmetycznych, wykonywanych na liczbach stałych. W przypadku wykorzystania \textit{.NET Framework}, obliczenia wykonywane są przy generowaniu kodu, a nie podczas uruchomienia skompilowanego programu. Przykładowo wykonanie działania $x = 1 + 2$, wygeneruje instrukcje jedynie do przypisania wartości $3$ do zmiennej $x$. W tworzonym kompilatorze, wygenerowanie zostanie instrukcji do załadowania liczby $1$ oraz $2$ na stos, następnie wykonanie operacji dodawania i na końcu przypisanie wyniku do zmiennej $x$.
\par Przy dodawaniu do siebie wielu łańcuchów znaków, w kodzie assemblera, programu napisanego w języku C\#, można zauważyć, że wszystkie łańcuchy są łączone przy pomocy jednej instrukcji \texttt{Concat}, gdzie w tworzonym kompilatorze, zawsze łączone są jedynie dwa łańcuchy na raz. Co więcej, można zauważyć też różnicę przy konwersji wartości zmiennych na łańcuchy znaków przy dodawaniu ich do elementów typu łańcuchowego. W tworzonym kompilatorze konwersja przebiega przy pomocy instrukcji \texttt{ToString()}, jednak w kompilatorze środowiska \textit{.NET Framework} przeprowadzana jest przy pomocy instrukcji \texttt{box}, która sprowadza zmienną do typu \texttt{object}. Instrukcja \texttt{Concat} przyjmuje wtedy elementy typu \texttt{object}.

\begin{lstlisting}[language=IL, caption={Fragment kodu deasemblerowanego testu programu C\#, przedstawiający łączenie łańcuchów znaków}, label=alg:asm]
  ...
  IL_006a:  ldloc.s    V_1
  IL_006c:  ldc.i4.2
  IL_006d:  box        [mscorlib]System.Int32
  IL_0072:  ldstr      "Tekst"
  IL_0077:  call       string [mscorlib]System.String::Concat(object,
                                                              object,
                                                              object)
  ...
\end{lstlisting}

\par Dalej analizując poszczególne moduły, można zauważyć, że w generowanym kodzie assemblera dla programów kompilowanych przy pomocy narzędzi środowiska \textit{.NET Framework}, naturalnie wykorzystywane są wszystkie dostępne instrukcje. Szczególnie widoczne jest to przy instrukcjach warunkowych. W tworzonym kompilatorze warunki zawsze są sprowadzane do pojedynczej wartości logicznej, a następnie na jej podstawie wykonywany jest skok. W przypadku kompilatora \textit{.NET}, przykładowo wykorzystywane są instrukcje które od razu porównują dwie wartości na jej podstawie wykonują skok.

\begin{lstlisting}[language=IL, caption={Fragment kodu deasemblerowanego testu programu C\#, przedstawiający instrukcje \texttt{if ... else}}, label=alg:asm]
  ...
  IL_0060:  ldc.i4.s   20
  IL_0062:  ldloc.s    number1
  IL_0064:  ble.s      IL_0072

  IL_0066:  ldstr      "number1 is less than 20"
  IL_006b:  call       void [mscorlib]System.Console::WriteLine(string)
  IL_0070:  br.s       IL_007c

  IL_0072:  ldstr      "number1 is greater than 20"
  IL_0077:  call       void [mscorlib]System.Console::WriteLine(string)
  IL_007c:  
  ...
\end{lstlisting}

\begin{lstlisting}[language=IL, caption={Fragment wygenerowanego kodu asemblerowanego z języka JavaScript, przedstawiający instrukcje \texttt{if ... else}}, label=alg:asm]
  ...
  ldc.i4 20
  ldloc v_number1
  cgt
  stloc v_tmp_11
  // BOOLEAN:tmp_11 = INTEGER:20 > INTEGER:number1
  
  ldloc v_tmp_11
  brfalse IF_2
  // if (BOOLEAN:tmp_11 == false) JUMP IF_2
  ldstr "number1 is less than 20"
  call void [mscorlib]System.Console::WriteLine(string)
  
  br IF_3
  IF_2: 
  // ELSE
  ldstr "number1 is greater than 20"
  call void [mscorlib]System.Console::WriteLine(string)
  ...
\end{lstlisting}

% w pętlach jest inna kolejność instrukcji
% \subsection{Porównanie zdekompilowanego kodu programu w postaci języka C\#}

\section{Testy algorytmów}

\par W celu przetestowania działania tworzonego kompilatora, zostały wykorzystane algorytmy odnalezione w Internecie. Zostaną dla nich utworzone odpowiednie kody w języku C\# w celu porównania szybkości działania, zużycia pamięci czy wykorzystania przestrzeni dyskowej. Dla kodu programu w JavaScript zostaną również utworzone wersje przystosowane do kompilatora JScript, którego wyniki również zostaną porównane.

\subsection{Algorytm sortowania}
\par Pierwszym z odnalezionych kodów jest algorytm sortowania bąbelkowego zamieszczony w artykule zamieszczonego pod adresem \url{https://www.educba.com/sorting-algorithms-in-javascript/}, którego autorem jest Priya Pedamkar. Kod przedstawia się następująco:

\begin{lstlisting}[language=JavaScript, caption={Algorytm sortowania bąbelkowego. Źródło: \url{https://www.educba.com/sorting-algorithms-in-javascript/}}, label=alg:alg1]
function swap(arr, firstIndex, secondIndex){
  var temp = arr[firstIndex];
  arr[firstIndex] = arr[secondIndex];
  arr[secondIndex] = temp;
}
function bubbleSortAlgo(arraaytest){
  var len = arraaytest.length,
  i, j, stop;
  for (i=0; i < len; i++){
    for (j=0, stop=len-i; j < stop; j++){
      if (arraaytest[j] > arraaytest[j+1]){
        swap(arraaytest, j, j+1);
      }
    }
  }
  return arraaytest;
}
console.log(bubbleSortAlgo([3, 6, 2, 5, -75, 4, 1]));
\end{lstlisting}

\par Niestety funkcjonalność obsługi tablic nie została w pełni zaimplementowana, co powoduje, że powyższy kod dla stworzonego kompilatora nie jest poprawny. Problem występuje w dwóch miejscach. Pierwszy z nich znajduje się w wewnętrznej instrukcji \texttt{for} a dokładniej w warunku kończącym iterowanie pętli. W języku JavaScript przy odwołaniu się do elementu poza zakresem tablicy, nie zostanie wywołany błąd a jedynie zostanie zwrócona wartość typu \texttt{undefined}. W środowisku \textit{.NET Framework} wyjście poza zakres powoduje błąd i zatrzymanie działania programu. W celu uniknięcia tego, zmienna \texttt{stop} została zmniejszona o $1$. 
\par Drugim miejscem jest wywołanie funkcji, podając dla niej bezpośrednio tablicę wartości. Problem polega na braku obsłużenia przedstawionej sytuacji, w której przekazywane jest bezpośrednio nowo utworzona tablica elementów, co skutkuje wygenerowaniem błędnego kodu asemblera, którego po skompilowaniu do postaci binarnej, uruchomienie powoduje błąd środowiska maszyny wirtualnej. Aby uniknąć tego błędu, nowo tworzona tablica będzie najpierw przypisana do zmiennej, a dopiero następnie przekazana do funkcji.

\begin{lstlisting}[language=JavaScript, caption={Zmodyfikowany kod programu sortowania bąbelkowego.}, label=alg:alg1]
  function swap(arr, firstIndex, secondIndex){
    var temp = arr[firstIndex];
    arr[firstIndex] = arr[secondIndex];
    arr[secondIndex] = temp;
  }
  function bubbleSortAlgo(arraaytest){
    var len = arraaytest.length,
    i, j, stop;
    for (i=0; i < len; i++){
      for (j=0, stop=len - i - 1; j < stop; j++){
        if (arraaytest[j] > arraaytest[j+1]){
          swap(arraaytest, j, j+1);
        }
      }
    }
    return arraaytest;
  }
  var l = [3, 6, 2, 5, -75, 4, 1];
  console.log(bubbleSortAlgo(l));
\end{lstlisting}



% \subsection{Algorytm wyszukiwania}



% 1. Obsługa standardowego wyjścia
% 2. Obsługa zmiennych
% 3. Obsługa działań arytmetycznych
% 4. Obsługa wyrażeń warunkowych
% 5. Obsługa pętli
% 6. Obsługa tablic
% 7. Obsługa funkcji
