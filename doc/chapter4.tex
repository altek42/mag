\chapter{Testy}
\label{rozdzial4}

\par Podczas procesu tworzenia kompilatora, były również tworzone programy testujące dla poszczególnych funkcjonalności, w celu zweryfikowania prawidłowego działania programu. Dla każdego z modułów został utworzony program testujący w języku JavaScript, który obejmuje zakres funkcjonalności modułu. Zostały również utworzone tożsame programy w języku C\# w celu porównania do kodu języka JavaScript.
\par Zostały również wykorzystane dwa gotowe programy napisane w JavaScript odnalezione w Internecie. Również dla tych programów zostały utworzone odpowiedniki w języku C\# oraz zostały stworzone programy w języku JScript. Dla programów zostały przeprowadzone dodatkowe testy: został zmierzony czas wykonywania programu, zużycie pamięci oraz wielkości pliku wynikowego.

\section{Porównanie wyniku wykonania programów}

\par A

% 1. Obsługa standardowego wyjścia
% 2. Obsługa zmiennych
% 3. Obsługa działań arytmetycznych
% 4. Obsługa wyrażeń warunkowych
% 5. Obsługa pętli
% 6. Obsługa tablic
% 7. Obsługa funkcji


% \section{Proste operacje matematyczne}
% Test dodawania, odejmowania, mnożenia, dzielenia, przypisywania.
% \section{Kolejność wykonywania działań}
% Test na bardziej złożonych wyrażeniach. Sprawdzenie poprawności działania nawiasów oraz kolejności wykonywania działań.
% \section{Wyrażenia warunkowe}
% Test wyrażeń warunkowych. Kolejność wykonywania operacji and i or.
% \section{Tablice}
% Test obsługi tablic jedno i wielowymiarowych.
% \section{Obiekty}
% Test obsługi obiektów.
% \section{Klasy}
% Jeśli będzie implementacja.
% Test działania obiektów klas.
% \section{Funkcje}
% Test działania funkcji.
% 1. Funkcja "void" bez parametrów.
% 2. Funkcja "void" z parametrami.
% 3. Funkcja zwracająca różne typy (proste, tablice, obiekty) bez parametrów.
% 4. Funkcje zwracająca różne typy z parametrami.
% 5. inne

% \section{Algorytm 1}
% \subsection{Opracowanie pseudokodu algorytmu 1}
% \subsection{Implementacja algorytmu 1}
% \subsection{Testy algorytmu 1}
% \section{Algorytm 2}
% \subsection{Opracowanie pseudokodu algorytmu 2}
% \subsection{Implementacja algorytmu 2}
% \subsection{Testy algorytmu 2}
