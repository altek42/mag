\chapter{Projekt kompilatora}
\label{rozdzial2}
Opis 
\section{Środowisko i narzędzia}
Opis sprzętu na którym będzie wszystko uruchomiane.
Do implementacji będą wykorzystane:

\begin{itemize}
  \item C\#
  \item Visual Studio Code
  \item WSL (Ubuntu 20.04 LTS)
  \item JavaScript
  \item Node.js
  \item mono-devel (ilasm, ildasm)
\end{itemize}

\section{Analiza języka JavaScript i określenie zakresu implementacji}
% Opis składni JavaScript.
% Implementacja JavaScript w standardzie ES5.
% Musi być Turing-complete. Dodatkowo implementacja funkcji.

\par W tym rozdziale zawarty zostanie zakres implementacji oraz opis poszczególnych elementów języka JavaScript. W projekcie zostanie zaimplementowana jedynie część standardu ECMAScript, a niektóre mechanizmy zostaną uproszczone.

% #PYTANIE Czy warto opisać rzeczy które nie będą implementowane?

\subsection{Wyrażenia}

\par Składnia języka JavaScript zapożycza wiele rozwiązań użytych w Javie, jednak na konstrukcję miły też wpływ takie języki jak: Awk, Perl i Python. W języku JavaScript instrukcję nazywane są wyrażeniami, które rozdzielane są znakiem średniaka. Znaki białe takie jak spacja, tabulator czy znak końca linii nie mają wpływu na sposób działania kolejnych elementów wyrażeń, stanowią jedynie sposób ich oddzielenia. W kodzie źródłowym JavaScript rozróżnialna jest wielkość liter oraz wspierany jest standard znaków Unicode. ECMAScript definiuje również zestaw słów kluczowych i literałów oraz zasady automatycznego umieszczania średników ASI (Automatic semicolon insertion).

\subsection{Komentarze}

\par Rozróżniane są dwa typy komentarzy:
\begin{enumerate}
  \item Jednoliniowy - definiowany jest przy pomocy znaku ``$//$'' oraz umieszczany jest na końcu linii.
  \begin{lstlisting}[language=JavaScript, caption=Przykład komentarza jednoliniowego, label=alg:kod1]
    console.log(); // komentarz
  \end{lstlisting}
  \item Wieloliniowy - zawarty jest pomiędzy dwoma elementami ``/*'' oraz ``*/''
  \begin{lstlisting}[language=JavaScript, caption=Przykład komentarza wieloliniowego, label=alg:kod2]
    console.log();
    /*
      komentarz na
      wiele linii
    */
  \end{lstlisting}
\end{enumerate}

\subsection{Deklaracje zmiennych i stałych}
\par Zmienne deklaruje się przy pomocy słów kluczowych \texttt{var}, \texttt{let} oraz \texttt{const}. Deklaracja przy pomocy \texttt{var} jest podstawowym sposobem tworzenia zmiennych w JavaScript. Zasięg takiej zmiennej nie może być ograniczony przez blok w którym jest zawarta, przez co może powodować błędy przy pisaniu kodu. W celu uściślenia zasięgu i przeznaczenia zmiennych powstały dwa inne sposoby deklaracji \texttt{let} oraz \texttt{const}. Oba te rodzaje deklaracji powodują, że zakres dostępności zmiennej jest ograniczony do bloku w którym została zadeklarowana. Różnicą między tymi dwoma deklaracjami jest taka, że przy pomocy \texttt{const} definiujemy stałą która musi być od razu zadeklarowana, a \texttt{let} działa podobnie jak \texttt{var}.
\begin{lstlisting}
  var zmienna1;
  let zmienna2;
  const stala = true;
\end{lstlisting}
\par Przy deklaracji zmiennych przy użyciu \texttt{var} lub \texttt{let}, których nie przypiszemy żadnej wartości to przyjmują one wartość \texttt{undefined}

\subsection{Typy danych}
\par W najnowszym standardzie ECMAScript zdefiniowanych jest siedem typów danych:
\begin{enumerate}
  \item \texttt{Boolean} - może przybierać dwie wartości \texttt{true} lub \texttt{false}.
  \item \texttt{null} - słowo kluczowe oznaczające wartość zerową. 
  \item \texttt{undefined} - wartość nieokreślona.
  \item \texttt{Number} - tym przeznaczony dla literałów całkowitych jak i zmiennoprzecinkowych.
  \item \texttt{String} - typ przeznaczony dla literałów łańcuchowych reprezentujących zero lub więcej pojedynczych znaków ujętych w podwójny lub pojedynczy cudzysłów.
  \item \texttt{Symbol} - wprowadzony w ECMAScript 6 typ danych, który pozwala na tworzenie unikalnych i nie zmiennych wartości.
  \item \texttt{Object} - typ złożony do którego zaliczają się funkcje, tablice, słowniki oraz instancje klas.
\end{enumerate}

\subsection{Operacje arytmetyczne}
\subsection{Operacje warunkowe}
\subsection{Zakres implementacji projektu}
W niniejszym projekcie zostaną zaimplementowane następujące elementy: 
\begin{itemize}
  \item Komentarze jednoliniowe oraz wieloliniowe.
  \item Proste typy danych: \texttt{Boolean}, \texttt{Number}, \texttt{String}.
  \item Tworzenie zmiennych typu \texttt{var}.
  \item Uproszczona implementacja funkcji \texttt{console.log()}.
  \item Konwersja typów danych.
  \item Operacje matematyczne takie jak dodawanie, odejmowanie, mnożenie oraz dzielenie.
  \item Wyrażenia warunkowe takie jak sprawdzenie: równości, nierówności, większości lub mniejszości.
  \item Instrukcja \texttt{if} oraz pętle \texttt{while} oraz \texttt{for}
  \item Typ \texttt{Object} pod postacią tablicy elementów oraz słownika danych.
  \item Deklaracja oraz wywoływanie funkcji: bez parametrów oraz zwracanej wartości, bez parametrów oraz z zwracaną wartością, z parametrami bez zwracanej wartości oraz z parametrami z zwracaną wartością.
\end{itemize}

% https://developer.mozilla.org/pl/docs/Web/JavaScript/Guide/Sk%C5%82adnia_i_typy

\section{Parser}
Używamy ANTLR z własną gramatyką ale posiłkując się gotowcem. Wykorzystanie gotowej gramatyki powoduje wygenerowanie tak dużych plików, że próba zrozumienia co jest do czego wymaga poświęcenia dużego wysiłku. Większość tych rzeczy i tak by nie została wykorzystana.

Rozważane możliwości i wykonano przegląd narzędzi:

po 2 zdania: \\
Gotowe narzędzia:
\begin{itemize}
  \item LEX \& YYAC
  \item ANTLR
  \item Coco/R
  \item gppg \& gplex
  \item Owl (https://github.com/ianh/owl)
\end{itemize}
i więcej... https://en.wikipedia.org/wiki/Comparison\_of\_parser\_generators


\section{Struktura projektu}
Diagramy i opisy.
Jak będzie wyglądał ten rozdział zależy jak wyjdzie implementacja.
% Tak, tak, wiem, najpierw implementacja a później dokumentacja...
