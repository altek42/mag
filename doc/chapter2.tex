\chapter{Projekt kompilatora}
\label{rozdzial2}
\section{Środowisko i narzędzia}
\par Kod kompilatora zostanie zrealizowany w języku \textit{C\#} na platformie \textit{.NET Core 3.1}. W projekcie zostanie również wykorzystana platforma \textit{.NET Framework} w celu dekompilacji skompilowanego kodu testów, ich kompilacji z kodu \textit{IL Assembler} oraz kompilacji kodu \textit{JScript}. W celu weryfikacji poprawności kodu \textit{JavaScript} zostanie wykorzystana platforma \textit{Node.js} w wersji 10.15. Zostanie również wykorzystany skrypt pomocniczy w \textit{PowerShell} służący do łatwego przeprowadzenia testów oraz do zebrania informacji o wielkości plików wykonywalnych i pomiaru czasu wykonania programów testowych. Drugim z dekompilatorów jaki zostanie wykorzystany w projekcie jest program \textit{JetBrains dotPeek 2020.2.1}. Ostatnim z programów, służący do pomiarów zużycia pamięci, jest \textit{JetBrains dotMemory 2020.2.1}.
\par Wszystkie testy oraz pomiary zostaną przeprowadzone na komputerze o następujących parametrach:
\begin{itemize}
  \item System operacyjny: Windows 10 Home
  \item Typ systemu: 64-bitowy
  \item Procesor: Intel(R) Core(TM) i7-4700MQ CPU @ 2.40Hz
  \item Pamięć RAM: 16 GB
\end{itemize}

\section{Analiza języka JavaScript i określenie zakresu implementacji}
% Opis składni JavaScript.
% Implementacja JavaScript w standardzie ES5.
% Musi być Turing-complete. Dodatkowo implementacja funkcji.

\par W tym rozdziale zawarty zostanie zakres implementacji oraz opis poszczególnych elementów języka JavaScript. W projekcie zostanie zaimplementowana jedynie część standardu ECMAScript, a niektóre mechanizmy zostaną uproszczone.

% #PYTANIE Czy warto opisać rzeczy które nie będą implementowane?

\subsection{Wyrażenia}

\par Składnia języka JavaScript zapożycza wiele rozwiązań użytych w Javie, jednak na konstrukcję miały też wpływ takie języki jak: Awk, Perl i Python. W języku JavaScript instrukcję nazywane są wyrażeniami, które rozdzielane są znakiem średniaka. Znaki białe takie jak spacja, tabulator czy znak końca linii nie mają wpływu na sposób działania kolejnych elementów wyrażeń, stanowią jedynie sposób ich oddzielenia. W kodzie źródłowym JavaScript rozróżnialna jest wielkość liter oraz wspierany jest standard znaków Unicode. ECMAScript definiuje również zestaw słów kluczowych i literałów oraz zasady automatycznego umieszczania średników ASI (Automatic semicolon insertion).

\subsection{Komentarze}

\par Rozróżniane są dwa typy komentarzy:
\begin{enumerate}
  \item Jednoliniowy - definiowany jest przy pomocy znaku ``$//$'' oraz umieszczany jest na końcu linii.
  \begin{lstlisting}[language=JavaScript, caption=Przykład komentarza jednoliniowego, label=alg:komentarze1]
    console.log(); // komentarz
  \end{lstlisting}
  \item Wieloliniowy - zawarty jest pomiędzy dwoma elementami ``/*'' oraz ``*/''
  \begin{lstlisting}[language=JavaScript, caption=Przykład komentarza wieloliniowego, label=alg:komentarze2]
    console.log();
    /*
      komentarz na
      wiele linii
    */
  \end{lstlisting}
\end{enumerate}

\subsection{Deklaracje zmiennych i stałych}
\par Zmienne deklaruje się przy pomocy słów kluczowych \texttt{var}, \texttt{let} oraz \texttt{const}. Deklaracja przy pomocy \texttt{var} jest podstawowym sposobem tworzenia zmiennych w JavaScript. Zasięg takiej zmiennej nie może być ograniczony przez blok w którym jest zawarta, przez co może powodować błędy przy pisaniu kodu. W celu uściślenia zasięgu i przeznaczenia zmiennych powstały dwa inne sposoby deklaracji \texttt{let} oraz \texttt{const}. Oba te rodzaje deklaracji powodują, że zakres dostępności zmiennej jest ograniczony do bloku w którym została zadeklarowana. Różnicą między tymi dwoma deklaracjami jest taka, że przy pomocy \texttt{const} definiujemy stałą która musi być od razu zadeklarowana, a \texttt{let} działa podobnie jak \texttt{var}.
\begin{lstlisting}[language=JavaScript, caption=Przykład deklaracji zmiennych, label=alg:deklaracjaZmiennej1]
  var zmienna1;
  let zmienna2;
  const stala = true;
\end{lstlisting}
\par Przy deklaracji zmiennych przy użyciu \texttt{var} lub \texttt{let}, dla których nie przypiszemy żadnej wartości, przyjmują one wartość \texttt{undefined}

\subsection{Typy danych}
% https://developer.mozilla.org/pl/docs/Web/JavaScript/Guide/Sk%C5%82adnia_i_typy
\par W najnowszym standardzie ECMAScript zdefiniowanych jest siedem typów danych:
\begin{enumerate}
  \item \texttt{Boolean} - może przybierać dwie wartości \texttt{true} lub \texttt{false}.
  \item \texttt{null} - słowo kluczowe oznaczające wartość zerową. 
  \item \texttt{undefined} - wartość nieokreślona.
  \item \texttt{Number} - tym przeznaczony dla literałów całkowitych jak i zmiennoprzecinkowych.
  \item \texttt{String} - typ przeznaczony dla literałów łańcuchowych reprezentujących zero lub więcej pojedynczych znaków ujętych w podwójny lub pojedynczy cudzysłów.
  \item \texttt{Symbol} - wprowadzony w ECMAScript 6 typ danych, który pozwala na tworzenie unikalnych i nie zmiennych wartości.
  \item \texttt{Object} - typ złożony do którego zaliczają się funkcje, tablice, słowniki oraz instancje klas.
\end{enumerate}

\subsection{Operacje arytmetyczne}
% https://developer.mozilla.org/en-US/docs/Web/JavaScript/Guide/Expressions_and_Operators
Operatory arytmetyczne przyjmują jako operandy wartości liczbowe w postaci literałów lub zmiennych i jako wynik zwracają pojedynczą wartość liczbową. Standardowe operatory arytmetyczne to:
\begin{itemize}
  \item dodawanie ``\texttt{+}''
  \item odejmowanie ``\texttt{-}''
  \item mnożenie ``\texttt{*}''
  \item dzielenie ``\texttt{/}''
\end{itemize}

\begin{lstlisting}[language=JavaScript, caption=Przykład użycia operatorów arytmetycznych, label=alg:operatoryArytmetyczne1]
  var v1 = 10 + 15
  var v2 = 100 + 100 / 2 * 3 + 10
  var v3 = 100.7 + 10.40 / 1.67 * 3.1 + 10.23
\end{lstlisting}


\subsection{Operacje porównania}
  \par Operatory porównania porównuje swoje operandy czego wynikiem jest wartość logiczna, określająca czy dane stwierdzenie jest prawdziwe. Operandy mogą być wartościami liczbowymi, łańcuchami znaków, logicznymi lub wartościami obiektu. W przypadku kiedy dwa operandy są różnego typu, JavaScript próbuje przekonwertować je na odpowiedni typ do porównania.
  \par Operatory porównania w języku JavaScript to:
  \begin{itemize}
    \item równość ``==''
    \item nierówność ``!=''
    \item ścisła równość ``===''
    \item ścisła nierówność ``!==''
    \item większy ``>''
    \item większy lub równy ``>=''
    \item mniejszy ``<''
    \item mniejszy lub równy ``<=''
  \end{itemize}
  Tutaj warto zauważyć, że JavaScript posiada dwa operatory równości i nierówności. Różnica pomiędzy ścisłym porównaniu a zwykłym polega na tym, że w przypadku kiedy operandy są różnego typu, to dla zwykłego porównywania, JavaScript próbuje przekonwertować wartości do jednego typu. Przy porównywaniu ścisłym nie zachodzi konwersja.

  \begin{lstlisting}[language=JavaScript, caption=Przykład użycia operatorów porównania, label=alg:operatoryPorownania1]
    var v1 = 2 == 2
    var v2 = 1 + 1 != 2
    var v3 = zmienna1 > zmienna2
  \end{lstlisting}

\subsection{Instrukcje warunkowe}
% https://developer.mozilla.org/pl/docs/Web/JavaScript/Guide/Control_flow_and_error_handling
Instrukcje warunkowe to zbiór instrukcji, których wykonanie jest zależne od zdefiniowanego warunku którego wynikiem jest wartość logiczna ``true'' lub ``false''. JavaScript wspiera dwa rodzaje instrukcji warunkowych:
\begin{itemize}
  \item \texttt{if .. else}
  \item \texttt{siwtch}
\end{itemize}
\par Instrukcja \texttt{if} wykonuje blok instrukcji w przypadku, kiedy podany warunek zwróci wartość ``true''. Jeśli trzeba obsłużyć przypadek kiedy warunek nie został spełniony, można posłużyć się instrukcją \texttt{else} lub instrukcją \texttt{else if} podając inny warunek.

\begin{lstlisting}[language=JavaScript, caption=Przykład użycia instrukcji \texttt{if .. else}, label=alg:instrukcjaIfElese1]
  if (20 > number1) {
    console.log("number1 is less than 20")
  } else if (25 > number1) {
    console.log("number1 is less than 25 and greater than 20")
  } else {
    console.log("number1 is greater than 25")
  }
\end{lstlisting}

\par Instrukcja \texttt{switch} wykonuje blok instrukcji w przypadku, kiedy podane wyrażenie zgadza się z identyfikatorom danego bloku. Po dopasowaniu identyfikatora, wykonywane są wszystkie bloki instrukcji poniżej dopasowania, chyba że zostanie użyte słowo kluczowe \texttt{break}, które zakańcza wykonywanie instrukcji switch. Jeśli dane wyrażenie nie zostanie dopasowane do żadnego z identyfikatora, wykonywany jest kod z bloku o identyfikatorze \texttt{default}.

\begin{lstlisting}[language=JavaScript, caption=Przykład użycia instrukcji \texttt{switch}, label=alg:instrukcjaSwitch1]
  switch (color) {
  case "Red":
    console.log("Chosen color: Red");
    break;
  case "Blue":
    console.log("Chosen color: Blue");
    break;
  case "Green":
    console.log("Chosen color: Green");
    break;
  default:
   console.log("Chosen color: Default");
}
\end{lstlisting}


\subsection{Pętle}
% https://developer.mozilla.org/pl/docs/Web/JavaScript/Guide/Loops_and_iteration
Przy pomocy pętli można w łatwy sposób powtarzać wykonywanie bloków instrukcji. W języku JavaScript występują różne rodzaje pętli. Można rozróżnić następujące konstrukcje:
\begin{itemize}
  \item \texttt{for}
  \item \texttt{for .. in}
  \item \texttt{for .. of}
  \item \texttt{do .. while}
  \item \texttt{while}
\end{itemize}

\par Pętla \texttt{for} przyjmuje jako parametry trzy elementy: wyrażenie inicjalizacji, warunek zakończenia oraz wyrażenie inkrementacji. Wyrażenie inicjalizacji wykonywane jest tylko raz, na samym początku, jeszcze przed sprawdzeniem instrukcji warunkowej. Zazwyczaj wykorzystuje się je do zadeklarowania lub wyzerowania zmiennej iterującej. Wyrażenie warunkowe sprawdzane jest przed każdym wywołaniem bloku instrukcji. W przypadku kiedy wyrażenie warunkowe jest prawdziwe to zostanie wykonany blok instrukcji, a jeśli jest fałszywe to zakończy się działanie pętli. Wyrażenie inkrementacji wywoływane jest po zakończeniu wykonywania bloku instrukcji. Wykorzystuje się je do modyfikacji zmiennej iterującej.

\begin{lstlisting}[language=JavaScript, caption=Przykład użycia instrukcji \texttt{for}, label=alg:instrukcjaFor1]
  for (var index = 0; index < 10; index = index + 1) {
    var element = x + index; // blok instrukcji
    console.log(element)
  }
\end{lstlisting}

\par Pętla \texttt{for .. in} jest instrukcją pozwalającą na przeiterowanie się po elementach obiektu. Przyjmuje jako parametry deklarację zmiennej, do której wpisywana będzie wartość iteratora w danym przebiegu pętli oraz jako drugi argument, podaje się obiekt po którym chcemy się przeiterować. 

\begin{lstlisting}[language=JavaScript, caption=Przykład użycia instrukcji \texttt{for .. in}, label=alg:instrukcjaFor2]
  for (var key in obj){
    console.log(obj[key])
  }
\end{lstlisting}

\par Pętla \texttt{for .. of} pozwala na iterowanie się po obiektach iterowalnych. Podobnie jak pętla \texttt{for .. in} przyjmuje dwa parametry, z których pierwszy również jest deklaracją zmiennej, jednak przechowuje ona wartość elementu w danej iteracji, a drugi parametr jest obiekt iterowalny.

\begin{lstlisting}[language=JavaScript, caption=Przykład użycia instrukcji \texttt{for .. of}, label=alg:instrukcjaFor3]
  for (var value of myArray){
    console.log(value)
  }
\end{lstlisting}

\par Pętla \texttt{do .. while} wykonuje blok instrukcji po którym sprawdzane jest wyrażenie warunkowe. Jeżeli wyrażenie jest prawdziwe wykonuje ponownie blok instrukcji, w przeciwnym wypadku wykonywanie pętli zostanie zakończone. Warto zwrócić uwagę, że blok instrukcji wykona się zawsze co najmniej raz.

\begin{lstlisting}[language=JavaScript, caption=Przykład użycia instrukcji \texttt{do .. while}, label=alg:instrukcjaWhile1]
  do {
    console.log(zmienna)
  } while (zmienna > 10)
\end{lstlisting}

\par Pętla \texttt{while} działa w taki sam sposób jak pętla \texttt{do .. while} z tą różnicą, że warunek jest sprawdzany przed wykonaniem bloku instrukcji. oznacza to, że jeśli w pierwszej iteracji warunek będzie fałszywy, to blok instrukcji nie wykona się ani razu.

\begin{lstlisting}[language=JavaScript, caption=Przykład użycia instrukcji \texttt{while}, label=alg:instrukcjaWhile2]
  while (zmienna > 10) {
    console.log(zmienna)
  }
\end{lstlisting}


\subsection{Tablice}
\par W języku JavaScript prawie wszystko jest obiektem. Każdy obiekt posiada swoje właściwości (atrybuty) oraz może posiadać swoje metody. Takimi obiektami mogą być między innymi takie konstrukcje jak tablice, obiekty słownikowe, funkcje czy też instancje klas.

\par Tablice (\texttt{Array}) pozwalają na przechowywanie grup danych i utrzymywaniu porządku w danych. Aby utworzyć obiekt tablicy można posłużyć się konstrukcją \texttt{new} lub też użyć nawiasów kwadratowych \texttt{[]}. 
% \footcite[39-40]{BeginningJavaScript}

\begin{lstlisting}[language=JavaScript, caption=Przykład utworzenia tablicy, label=alg:array]
  // Wykorzystanie operatora 'new'
  var tablica1 = new Array();
  // Wykorzystanie nawiasów
  var tablica2 = [];
\end{lstlisting}

\par Tablice posiadają metody pozwalające na zarządzanie nimi. Przykładowo aby przypisać wartości można użyć metody \texttt{push}, wtedy dodawany element zapisywany jest na ostatnie miejsce w tablicy, a żeby pobrać ostatni element można użyć metody \texttt{pop}.

\begin{lstlisting}[language=JavaScript, caption=Przykład zarządania elementami tablicy przy pomocy `push' i `pop', label=alg:array2]
  var tablica = []
  // Dodanie elementu przy pomocy 'push'
  tablica.push('element')
  // Pobranie elementu przy pomocy 'pop'
  var element = tablica.pop()
\end{lstlisting}

\par Przypisywanie oraz pobieranie elementów można również wykonywać za pomocą indeksów. Elementy tablic numerowane są od zera.

\begin{lstlisting}[language=JavaScript, caption=Przykład zarządania elementami tablicy przy pomocy nawiasów, label=alg:array3]
  var tablica = []
  // Dodanie elementu
  tablica[0] = 'element'
  // Pobranie elementu
  var element = tablica[0]
\end{lstlisting}

\par Tablice są obiektami dynamicznymi i w każdej chwili można zmienić ilość znajdujących się w nich elementów. W związku z tym tablica posiada atrybut \texttt{length} przechowujący ilość elementów znajdujący się w danej tej tablicy. 

\begin{lstlisting}[language=JavaScript, caption=Przykład pobrania ilości elementów tablicy , label=alg:array4]
  var tablica = []
  tablica[0] = 'element1'
  tablica[1] = 'element2'
  // Pobranie ilości elementów
  var len = tablica.length
\end{lstlisting}

\par Warto również wspomnieć, że tablice pozwalają na przechowywanie elementów o różnych typach, jak i mogą posiadać elementy puste.

\begin{lstlisting}[language=JavaScript, caption=Przykład przypisania elementów o różnych typach , label=alg:array5]
  var tablica = []
  tablica[0] = 'element1'
  tablica[1] = 2
  tablica[3] = true
  var len = tablica.length 
  // tablica zawiera 4 elementy
  // element o indeksie 2 jest pusty
\end{lstlisting}

\subsection{Funkcje}

\par Funkcja jest zbiorem instrukcji, które wykonują określone zadanie. Przed użyciem funkcji trzeba najpierw ją zdefiniować. Aby to zrobić należy użyć instrukcji \texttt{function}. Instrukcja ta wymaga podania: nazwy funkcji, listy argumentów oraz blok instrukcji, które mają być wykonywane w ramach tejże funkcji.

\begin{lstlisting}[language=JavaScript, caption=Przykład deklaracji funkcji, label=alg:function1]
  function square(number) {
    return number * number;
  }
\end{lstlisting}

\par Aby wywołać funkcję podać jej nazwę oraz listę argumentów. Argumenty mogą być dowolnego typu, jednak trzeba pamiętać, że przekazywanie typów prostych takich jak liczby lub ciągi znaków, przekazywane są przez wartości. Natomiast przekazywanie obiektów, takich jak tablice, odbywa się poprzez referencje do danego obiektu. Oznacza to, że zmiana w obiekcie spowoduje zmianę obiektu poza funkcją.

\begin{lstlisting}[language=JavaScript, caption=Przykład uzycia funkcji, label=alg:function2]
  var result = square(5)
\end{lstlisting}

\subsection{Zakres implementacji projektu}
W niniejszym projekcie zostaną zaimplementowane następujące elementy: 
\begin{itemize}
  \item Komentarze jednoliniowe oraz wieloliniowe
  \item Proste typy danych: 
  \begin{itemize}
    \item \texttt{Boolean} - wartości \textit{true} oraz \textit{false}
    \item \texttt{Number} - wartości całkowite oraz rzeczywiste
    \item \texttt{String} - łańcuchy znaków
  \end{itemize}
  \item Tworzenie zmiennych typu \texttt{var}, z opcją deklaracji kolejnych zmiennych po przecinku
  \item Uproszczona implementacja funkcji \texttt{console.log()}
  \item Konwersja typów danych
  \item Operacje matematyczne takie jak dodawanie, odejmowanie, mnożenie oraz dzielenie
  \item Konkatenacja łańcuchów znaków
  \item Operacje porównania
  \item Negacja wartości \texttt{Boolean}
  \item Instrukcja warunkowa \texttt{if}
  \item Pętle \texttt{while} oraz \texttt{for}
  \item Uproszczona tablica elementów - przechowywanie tylko wartości liczbowych całkowitych
  \item Deklaracja oraz wywoływanie funkcji
\end{itemize}


\section{Parser}
\par Do celów analizy składni języka JavaScript zostanie wykorzystane narzędzie ANTLR, które dostępne jest na wielu platformach oraz dla wielu języków. Wraz z narzędziem dostępna jest gotowa gramatyka dla języka JavaScript. Jednak, mimo dostępności pełnej gramatyki, zostanie na jej podstawie na nowo, tylko w obrębie planowanego zakresu. Dzięki temu zostaną uniknięte błędy, związane z brakiem pełnej implementacji funkcjonalności, oraz pozwoli to na stworzenie uproszczeń implementacyjnych.

\par Przy wyborze parsera, były również rozważane następujące narzędzia:
\begin{itemize}
  \item LEX \& YACC - standardowe narzędzia w systemach UNIX pozwalające na analizę plików tekstowych oraz generowanie na ich podstawie kodu w języku C.
  \item Coco / R - generator kompilatorów, który dla podanej gramatyki kodu źródłowego generuje parser dla tego języka. Implementacja dostępna w językach C\#, Java, C++ i innych.
  \item gppg \& gplex - implementacja LEX \& YACC w języku C\#.
  \item Owl - generator parserów, generujący do użycia plik nagłówkowy w języku C.
\end{itemize}
% i więcej... https://en.wikipedia.org/wiki/Comparison\_of\_parser\_generators

% LEX \& YACC https://tldp.org/HOWTO/Lex-YACC-HOWTO.html
% Coco / R https://ssw.jku.at/Research/Projects/Coco/Doc/UserManual.pdf
% gppg https://github.com/k-john-gough/gppg
% Owl https://github.com/ianh/owl


\section{Struktura projektu}
Diagramy i opisy.
Jak będzie wyglądał ten rozdział zależy jak wyjdzie implementacja.
% Tak, tak, wiem, najpierw implementacja a później dokumentacja...
