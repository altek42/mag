\documentclass[a4paper]{article}

\usepackage[utf8]{inputenc}
\usepackage{polski}
\usepackage[polish]{babel}
\usepackage{graphicx}
\usepackage[backend=bibtex,style=verbose-trad2]{biblatex}


\bibliography{bib}
\graphicspath{ {images/} }

\setcounter{page}{3}


\begin{document}
\section*{Streszczenie pracy}
\addcontentsline{toc}{section}{\protect Streszczenie pracy}%
Celem niniejszej pracy dyplomowej...

\hspace{0pt}
\vfill

\section*{Abstract}
\addcontentsline{toc}{section}{\protect Abstract}%
The aim of this diploma thesis was...
\vfill
\hspace{0pt}

\newpage
\tableofcontents
\newpage

\section{Wstęp}
Motywacja, założenia i cele pracy.

\section{Teoria}
\par
Nie bardzo mam pomysł jak bym mógł nazwać ten rozdział. Chciałbym tu opisać wszystkie rzeczy wokół których będzie się obracać ta magisterka.

\textbf{To raczej we wstępie, drugi rozdział to z reguły (szczególnie w mgr) analiza istniejących rozwiązań. Tu warto opisać również pochodne rozwiązania (np. warto wspomnieć asm.js i WebAssembly oraz Emscripten jako rozwiązanie robiące coś w zasadzie odwrotnego), na pewno warto napisać o TypeScript, przede wszystkim node.js, może są jakieś inne rozwiązania kompilujące js.
Te wszystkie punkty, które Pan wypisał są istotne, po dodaniu tych kilku wspomnianych elementów ten rozdział może wyjśc całkiem spory.}

\subsection{JavaScript}
Opis ogólny
\subsection{Node.js}
Opis
\subsection{.NET}
Opis
\subsection{IL Assembler}
Opis
\subsection{JScript .NET}
Jest to implementacja JavaScript przez Microsoft. Chciałbym opisać tu różnice.
\subsection{Kompilator}
Opis

\section{Projekt kompilatora}
Opis 
\subsection{Narzędzia}
Opis narzędzia jakie będą wykorzystywane podczas implementacji. 
\\
Nie bardzo wiem czy to ma znaczenie w jakim języku będzie to pisane, ale zastanawiam się nad \textit{c++} albo \textit{c\#}.
\\
Pytanie, czy powinienem używać narzędzi takich jak Lex i YYAC. Myślę, że można by napisać własny prostszy parser, ale nie wiem jak by to czasowo wyglądało.

\textbf{Można użyć flexa i bisona, ale można skorzystać np. z antlr, które ma gotowe gramatyki dla różnych języków. Język implementacji nie jest istotny.}

\subsection{Analiza języka JavaScript i określenie zakresu implementacji}
Opis składni JavaScript

\textbf{Tu głównym elementem powinno być właśnie określenie zakresu implementacji.}

\subsection{?Struktura projektu}
Nie wiem czy konieczne.
\\
Opis i diagram klas?

\textbf{Warto wrzucić kilka prostych diagramów (ja osobiście nie przepadam za UML w pracach, ale są tacy recenzenci, którzy lubią).}

\section{Implementacja aplikacji kompilatora}
\subsection{?Parser}
Jeżeli będzie wykorzystany Lex i YYAC to rozdział nie będzie potrzebny.
\textbf{Tu nawet przy użyciu gotowych narzędzi warto napisać jedną stronę o sposobie ich użycia.}

\subsection{Analiza leksykalna}
Tekst
\subsection{Gramatyka}
Tekst
\subsection{Funkcjonalności}
Opis sposobu przetwarzania instrukcji
\subsection{Generowanie assemblera}
Tekst

\section{Algorytmy testujące}
Będę chciał przetestować dwa algorytmy
\\
Na tę chwilę muszę się jeszcze zastanowić jakie to mogły by być, aby nie były jakoś skomplikowane ale żeby coś robiły więcej niż wyświetlanie 2+2 na konsoli.

\textbf{Tu warto wykonać i opisać w krótkiej formie również proste testy. Moim zdaniem (proszę jednak pamiętać, że to Pan jest Autorem pracy i Pan podejmuje ostateczne decyzje) warto pokazać najpierw działanie prostych testów:}

\begin{itemize}
  \item \textbf{obliczenie prostego wyrażenia,}
  \item \textbf{obsługa warunków,}
  \item \textbf{obsługa pętli,}
  \item \textbf{obsługa tablic,}
  \item \textbf{obsługa struktur,}
  \item \textbf{tworzenie i obsługa obiektów,}
  \item \textbf{obsługa funkcji}
\end{itemize}
\textbf{Dopiero później powinny się znaleźć jakieś 2 bardziej złożone algorytmy}

\subsection{Algorytm 1}
\subsubsection{Opracowanie pseudokodu algorytmu 1}
\subsubsection{Implementacja algorytmu 1}
\subsection{Algorytm 2}
\subsubsection{Opracowanie pseudokodu algorytmu 2}
\subsubsection{Implementacja algorytmu 2}

\section{Testy działania algorytmów}
Sprawdzenie czasu wykonywania algorytmów, przy różnych rozmiarach problemu.
\\\\
Czy można tu jeszcze jakieś inne testy zrobić?

\textbf{ To raczej w jednym ciągu z punktem 5, dobra praca mgr powinna zawierać ten punkt całkiem spory.
Inne testy to pokazanie wygenerowanego kodu, ciekawy testem może być disassemblacja do kodu C\# przy pomocy jakiegoś narzędzia (np. dotPeek). Ponadto można sprawdzić (przy dużych rozmiarach danych użytych w algorytmie) zajętość pamięci.}

\section{Podsumowanie}

\end{document}
