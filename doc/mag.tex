\documentclass[a4paper]{article}

\usepackage[utf8]{inputenc}
\usepackage{polski}
\usepackage[polish]{babel}
\usepackage{graphicx}
\usepackage[backend=bibtex,style=verbose-trad2]{biblatex}


\bibliography{bib}
\graphicspath{ {images/} }

\setcounter{page}{3}


\begin{document}
\section*{Streszczenie pracy}
\addcontentsline{toc}{section}{\protect Streszczenie pracy}%
Celem niniejszej pracy dyplomowej...
W skrócie: Stworzyć kompilator który zamieni JavaScript na IL Assembler i będzie można go uruchomić na .NET. Następnie porównanie działania skryptów uruchamianych na maszynach Node.js oraz .NET. Można jeszcze porównać program napisany w C\# z programem napisanym w js.

\hspace{0pt}
\vfill

\section*{Abstract}
\addcontentsline{toc}{section}{\protect Abstract}%
The aim of this diploma thesis was...
\vfill
\hspace{0pt}

\newpage
\tableofcontents
\newpage

% Projekt i implementacja kompilatora języka JavaScript na platformę .NET.

\section{Wstęp}

\par W branży programistycznej istnieje bardzo dużo języków programowania oraz różnych środowisk uruchomieniowych dla nich przeznaczonych. Podczas tworzenia oprogramowania niezbędny jest dla programisty kompilator. Zamienia on kod programu napisanego w konkretnym języku programowania, na zrozumiały dla środowiska uruchomieniowego ciąg instrukcji. Warto podkreślić, że języki programowania, które są proste do zrozumienia i opanowania dla programistów oraz pozwalają na pisanie programów, które można uruchomić na różnych urządzeniach i umożliwiają tworzenie bardzo zróżnicowanych rodzajów oprogramowania, są o wiele częściej używane niż inne. Powoduje to, że powstaje dużo różnych bibliotek i frameworków ułatwiających i automatyzujących tworzenie oprogramowania.

\par Jednym z popularnych języków wśród programistów jest język JavaScript, który może być uruchomiany w przeglądarkach internetowych lub w maszynie wirtualnej takiej jak Node.js. Innym z popularnych środowisk uruchomieniowych jest platforma .NET, dla której istnieje wiele kompilatorów różnych języków. Wykorzystanie istniejących bibliotek, frameworków czy modułów napisanych w języku JavaScript w niezmienionej formie na platformie .NET, umożliwi programistom na tworzenie bardziej uniwersalnego kodu.

\par Celem niniejszej pracy jest zaprojektowanie oraz implementacja kompilatora języka JavaScript na kod IL Assembler uruchamiany na platformie .NET. Następnie w celu sprawdzenia poprawności działania kompilatora, zostaną przeprowadzone testy przy pomocy prostych implementacji kodu JavaScript. Zostaną również zaprojektowane oraz zaimplementowane dwa bardziej skomplikowane testy, do porównania maszyn wirtualnych Node.js oraz .NET. Zostanie także porównany kod assemblera generowanego przez kompilator .Net Core języka C\# z kodem generowanym przez implementowany kompilator.

\par Praca podzielona jest na cztery części. Pierwsza część opisuje pojęcia i technologie wykorzystywane do realizacji tego projektu oraz technologie pokrewne, które w pewnym stopniu realizują cel pracy lub realizują podobne założenia. Druga część poświęcona jest zaprojektowaniu tworzonego kompilatora. Kolejna część opisuje sposób implementacji tego kompilatora na podstawie zdefiniowanych założeń. Ostatnia część opisuje przeprowadzone testy realizowane w ramach pracy oraz przedstawia wyniki ich działania.

\newpage

\section{Pojęcia i technologie}

\subsection{Zakres projektu}
Wstęp do opisu technologii które będą wykorzystywane w projekcie.
\subsubsection{Kompilator}
Opis co to jest i do czego służy.
\subsubsection{JavaScript}
Opis
%

\subsubsection{Node.js}
Opis
\subsubsection{.NET Core}
Opis
\subsubsection{IL Assembler}
Opis

\subsection{Technologie pokrewne}
W tej sekcji będą przedstawione technologie pokrewne.
Najpierw technologie konkurencyjne, następnie istniejące rozwiązania a na końcu rozwiązania podobne/nawiązujące w jakiś sposób do tematu.

\subsubsection{ECMAScript}
W sumie jest to standard języka na podstawie którego definiowana jest składnia JavaScript. Czy warto o tym wspominać?

\subsubsection{ActionScript}
Obiektowy język oparty na ECMAScript używany w Adobe Flash/

\subsubsection{JScript}
Jest to implementacja JavaScript przez Microsoft która jest uruchomiana w środowisku .NET.
Istnieją pewne różnice w porównaniu do JavaScript.

\subsubsection{TypeScript}
Język programowania stworzony przez Microsoft. Uruchamiany na Node.js lub Deno. Może być przekompilowany do js es5 przez Babel.

\subsubsection{?CoffeeScript}
Język programowania kompilowany do JavaScript.
Nie wiem czy warto wspominać.

Inne języki kompilowane do JavaScript: Roy, Kaffeine, Clojure, Opal

\subsubsection{Babel}
Kompiluje przykładowo TypeScript do JavaScript lub JavaScript ES6 do ES5.

% Maszyna inna niż Node.js
\subsubsection{Deno}
Maszyna wirtualna dla języka JavaScript oraz TypeScript.
% Nie jestem pewien czy można na "żywca" uruchamiać na niej TypeScript. (Temat do sprawdzenia)

\subsubsection{asm.js}
Jeśli dobrze zrozumiałem jest to okrojony JavaScript który tak samo może być uruchamiany w przeglądarkach czy Node.js/Deno.
Wykorzystywany przy kompilacji kodu c++ do uruchamiania na tych maszynach.

\subsubsection{Emscripten}
% https://pl.wikipedia.org/wiki/Emscripten
Kompilator kodu LLVM do JavaScript.

\subsubsection{WebAssembly}
% https://devenv.pl/webassembly-nadciaga-rewolucja/
Język niskopoziomowy, który działa z szybkością zbliżoną do rozwiązań natywnych i pozwala na kompilację kodu napisanego w C/C++ do kodu binarnego działającego w przeglądarce internetowej.
(Pomija JavaScript).

\subsubsection{C\#}
Czy opisywać rodzinę .NET?

Na maszynę .NET istnieją implementacje języków takich jak Python, Java, C++ i inne.

\subsubsection{.NET Framework}
Platforma programistyczna.

\subsubsection{Mono}
Implementacja open source platformy .NET Framework.

\subsubsection{DotGNU}
Alternatywa dla Mono.

\subsubsection{?Roslyn}
.NET Compiler Platform


\section{Projekt kompilatora}
Opis 
\subsection{Środowisko i narzędzia}
Opis sprzętu na którym będzie wszystko uruchomiane.
Do implementacji będą wykorzystane:

\begin{itemize}
  \item C\#
  \item Visual Studio Code
  \item WSL (Ubuntu 20.04 LTS)
  \item JavaScript
  \item Node.js
\end{itemize}

\subsection{Analiza języka JavaScript i określenie zakresu implementacji}
Opis składni JavaScript.
Implementacja JavaScript w standardzie ES5.
Musi być Turing-complete. Dodatkowo implementacja funkcji.

\subsection{Parser}
Używamy ANTLR z gotową gramatyką

Rozważane możliwości i wykonano przegląd narzędzi:
Opcje:
1. Napiszę własną implementację, która może być mało czytelna i mieć dużo błędów. (Baza: http://informatyka.wroc.pl/node/391)
2. Użyję ANTLR i napiszę własną gramatykę.
3. Użyję ANTLR i użyję gotowe gramatyki.

po 2 zdania: 
Gotowe narzędzia:
\begin{itemize}
  \item LEX \& YYAC
  \item ANTLR
  \item Coco/R
  \item ?gppg \& gplex
  \item Owl (https://github.com/ianh/owl)
\end{itemize}
i więcej... https://en.wikipedia.org/wiki/Comparison\_of\_parser\_generators


\subsection{Struktura projektu}
Diagramy i opisy.
Jak będzie wyglądał ten rozdział zależy jak wyjdzie implementacja.
% Tak, tak, wiem, najpierw implementacja a później dokumentacja...

\section{Implementacja aplikacji kompilatora}
\subsection{Parser}
Sposób implementacji lub przygotowania i użycia gotowych narzędzi.

\subsection{Analiza leksykalna}
Tekst
\subsection{Gramatyka}
Tekst
\subsection{Funkcjonalności}
Opis sposobu przetwarzania instrukcji
\subsection{Generowanie assemblera}
Opisać jak wygenerować kod .NET

\section{Testy}
Opis zakresu testów i jak będą przebiegać.
Każdy z poniższych testów będzie sprawdzany pod kątem poprawności wykonywania działań, czasu wykonywania w porównaniu do wykonania kodu na Node.js (dla bardziej złożonych testów z czasem będzie więcej), zajętość pamięci (również tylko przy tych których to ma sens) oraz porównaniu kodu z asemblerowego wygenerowanego za pośrednictwem języka C\#.

\subsection{Proste operacje matematyczne}
Test dodawania, odejmowania, mnożenia, dzielenia, przypisywania.
\subsection{Kolejność wykonywania działań}
Test na bardziej złożonych wyrażeniach. Sprawdzenie poprawności działania nawiasów oraz kolejności wykonywania działań.
\subsection{Wyrażenia warunkowe}
Test wyrażeń warunkowych. Kolejność wykonywania operacji and i or.
\subsection{Tablice}
Test obsługi tablic jedno i wielowymiarowych.
\subsection{Obiekty}
Test obsługi obiektów.
\subsection{Klasy}
Jeśli będzie implementacja.
Test działania obiektów klas.
\subsection{Funkcje}
Test działania funkcji.
1. Funkcja "void" bez parametrów.
2. Funkcja "void" z parametrami.
3. Funkcja zwracająca różne typy (proste, tablice, obiekty) bez parametrów.
4. Funkcje zwracająca różne typy z parametrami.
5. inne

\subsection{Algorytm 1}
\subsubsection{Opracowanie pseudokodu algorytmu 1}
\subsubsection{Implementacja algorytmu 1}
\subsubsection{Testy algorytmu 1}
\subsection{Algorytm 2}
\subsubsection{Opracowanie pseudokodu algorytmu 2}
\subsubsection{Implementacja algorytmu 2}
\subsubsection{Testy algorytmu 2}

% Inne testy to pokazanie wygenerowanego kodu, ciekawy testem może być disassemblacja do kodu C\# przy pomocy jakiegoś narzędzia (np. dotPeek). Ponadto można sprawdzić (przy dużych rozmiarach danych użytych w algorytmie) zajętość pamięci.

\section{Podsumowanie}

\end{document}
