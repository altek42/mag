\documentclass[a4paper]{article}

\usepackage[utf8]{inputenc}
\usepackage{polski}
\usepackage[polish]{babel}
\usepackage{graphicx}
\usepackage[backend=bibtex,style=verbose-trad2]{biblatex}


\bibliography{bib}
\graphicspath{ {images/} }

\setcounter{page}{3}


\begin{document}
\section*{Streszczenie pracy}
\addcontentsline{toc}{section}{\protect Streszczenie pracy}%
Celem niniejszej pracy dyplomowej...
W skrócie: Stworzyć kompilator który zamieni JavaScript na IL Assembler i będzie można go uruchomić na .NET. Następnie porównanie działania skryptów uruchamianych na maszynach Node.js oraz .NET. Można jeszcze porównać program napisany w C\# z programem napisanym w js.

\hspace{0pt}
\vfill

\section*{Abstract}
\addcontentsline{toc}{section}{\protect Abstract}%
The aim of this diploma thesis was...
\vfill
\hspace{0pt}

\newpage
\tableofcontents
\newpage

\section{Wstęp}
Motywacja, założenia i cele pracy.
Nie więcej niż jedna strona.

\section{Pojęcia i technologie}
Wstęp do opisu technologii które będą wykorzystywane w projekcie.
\subsection{Kompilator}
Opis co to jest i do czego służy.
\subsection{JavaScript}
Opis
\subsection{Node.js}
Opis
\subsection{.NET Core}
Opis
\subsection{IL Assembler}
Opis

\section{Technologie pokrewne}
W tej sekcji będą przedstawione technologie pokrewne.
Najpierw technologie konkurencyjne, następnie istniejące rozwiązania a na końcu rozwiązania podobne/nawiązujące w jakiś sposób do tematu.

\subsection{ECMAScript}
W sumie jest to standard języka na podstawie którego definiowana jest składnia JavaScript. Czy warto o tym wspominać?

\subsection{ActionScript}
Obiektowy język oparty na ECMAScript używany w Adobe Flash/

\subsection{JScript}
Jest to implementacja JavaScript przez Microsoft która jest uruchomiana w środowisku .NET.
Istnieją pewne różnice w porównaniu do JavaScript.

\subsection{TypeScript}
Język programowania stworzony przez Microsoft. Uruchamiany na Node.js lub Deno. Może być przekompilowany do js es5 przez Babel.

\subsection{?CoffeeScript}
Język programowania kompilowany do JavaScript.
Nie wiem czy warto wspominać.

Inne języki kompilowane do JavaScript: Roy, Kaffeine, Clojure, Opal

\subsection{Babel}
Kompiluje przykładowo TypeScript do JavaScript lub JavaScript ES6 do ES5.

% Maszyna inna niż Node.js
\subsection{Deno}
Maszyna wirtualna dla języka JavaScript oraz TypeScript.
% Nie jestem pewien czy można na "żywca" uruchamiać na niej TypeScript. (Temat do sprawdzenia)

\subsection{asm.js}
Jeśli dobrze zrozumiałem jest to okrojony JavaScript który tak samo może być uruchamiany w przeglądarkach czy Node.js/Deno.
Wykorzystywany przy kompilacji kodu c++ do uruchamiania na tych maszynach.

\subsection{Emscripten}
% https://pl.wikipedia.org/wiki/Emscripten
Kompilator kodu LLVM do JavaScript.

\subsection{WebAssembly}
% https://devenv.pl/webassembly-nadciaga-rewolucja/
Język niskopoziomowy, który działa z szybkością zbliżoną do rozwiązań natywnych i pozwala na kompilację kodu napisanego w C/C++ do kodu binarnego działającego w przeglądarce internetowej.
(Pomija JavaScript).

\subsection{C\#}
Czy opisywać rodzinę .NET?

Na maszynę .NET istnieją implementacje języków takich jak Python, Java, C++ i inne.

\subsection{.NET Framework}
Platforma programistyczna.

\subsection{Mono}
Implementacja open source platformy .NET Framework.

\subsection{DotGNU}
Alternatywa dla Mono.



\section{Projekt kompilatora}
Opis 
\subsection{Środowisko i narzędzia}
Opis sprzętu na którym będzie wszystko uruchomiane.
Do implementacji będą wykorzystane:

\begin{itemize}
  \item C\#
  \item Visual Studio Code
  \item WSL (Ubuntu 20.04 LTS)
  \item JavaScript
  \item Node.js
\end{itemize}

\subsection{Analiza języka JavaScript i określenie zakresu implementacji}
Opis składni JavaScript.
Implementacja JavaScript w standardzie ES5.
Musi być Turing-complete. Dodatkowo implementacja funkcji.

\subsection{Parser}
Opcje:
1. Napiszę własną implementację, która może być mało czytelna i mieć dużo błędów. (Baza: http://informatyka.wroc.pl/node/391)
2. Użyję ANTLR i napiszę własną gramatykę.
3. Użyję ANTLR i użyję gotowe gramatyki.

Gotowe narzędzia:
\begin{itemize}
  \item LEX \& YYAC
  \item ANTLR
  \item Coco/R
  \item ?gppg \& gplex
  \item Owl (https://github.com/ianh/owl)
\end{itemize}
i więcej... https://en.wikipedia.org/wiki/Comparison\_of\_parser\_generators


\subsection{Struktura projektu}
Diagramy i opisy.
Jak będzie wyglądał ten rozdział zależy jak wyjdzie implementacja.
% Tak, tak, wiem, najpierw implementacja a później dokumentacja...

\section{Implementacja aplikacji kompilatora}
\subsection{Parser}
Sposób implementacji lub przygotowania i użycia gotowych narzędzi.

\subsection{Analiza leksykalna}
Tekst
\subsection{Gramatyka}
Tekst
\subsection{Funkcjonalności}
Opis sposobu przetwarzania instrukcji
\subsection{Generowanie assemblera}
Tekst

\section{Testy}
Opis zakresu testów i jak będą przebiegać.
Każdy z poniższych testów będzie sprawdzany pod kątem poprawności wykonywania działań, czasu wykonywania w porównaniu do wykonania kodu na Node.js (dla bardziej złożonych testów z czasem będzie więcej), zajętość pamięci (również tylko przy tych których to ma sens) oraz porównaniu kodu z asemblerowego wygenerowanego za pośrednictwem języka C\#.

\subsection{Proste operacje matematyczne}
Test dodawania, odejmowania, mnożenia, dzielenia, przypisywania.
\subsection{Kolejność wykonywania działań}
Test na bardziej złożonych wyrażeniach. Sprawdzenie poprawności działania nawiasów oraz kolejności wykonywania działań.
\subsection{Wyrażenia warunkowe}
Test wyrażeń warunkowych. Kolejność wykonywania operacji and i or.
\subsection{Tablice}
Test obsługi tablic jedno i wielowymiarowych.
\subsection{Obiekty}
Test obsługi obiektów.
\subsection{Klasy}
Jeśli będzie implementacja.
Test działania obiektów klas.
\subsection{Funkcje}
Test działania funkcji.
1. Funkcja "void" bez parametrów.
2. Funkcja "void" z parametrami.
3. Funkcja zwracająca różne typy (proste, tablice, obiekty) bez parametrów.
4. Funkcje zwracająca różne typy z parametrami.
5. inne

\subsection{Algorytm 1}
\subsubsection{Opracowanie pseudokodu algorytmu 1}
\subsubsection{Implementacja algorytmu 1}
\subsubsection{Testy algorytmu 1}
\subsection{Algorytm 2}
\subsubsection{Opracowanie pseudokodu algorytmu 2}
\subsubsection{Implementacja algorytmu 2}
\subsubsection{Testy algorytmu 2}

% Inne testy to pokazanie wygenerowanego kodu, ciekawy testem może być disassemblacja do kodu C\# przy pomocy jakiegoś narzędzia (np. dotPeek). Ponadto można sprawdzić (przy dużych rozmiarach danych użytych w algorytmie) zajętość pamięci.

\section{Podsumowanie}

\end{document}
