%%%%%%%%%%%%%%%%%%%%%%%%%%%%%%%%%%%%%%%%%
% Główny plik pracy
% Szablon pracy dyplomowej
% Wydział Informatyki 
% Zachodniopomorski Uniwersytet Technologiczny w Szczecinie
% autor Joanna Kołodziejczyk (jkolodziejczyk@zut.edu.pl)
% Bardzo wczesnym pierwowzorem szablonu był
% The Legrand Orange Book
% Version 2.1 (26/09/2018)
%
% Modifications to LOB assigned by %JK
%%%%%%%%%%%%%%%%%%%%%%%%%%%%%%%%%%%%%%%%%


%----------------------------------------------------------------------------------------
%	PAKIETY ORAZ PLIKI ZAWIERAJĄCE DEFINICJE STYLI I KONFIGURACJĘ OTOCZEŃ LATEX
%----------------------------------------------------------------------------------------

\documentclass[12pt,fleqn,twoside]{book} % JK Rozmiary czcionek są zgodne z wymaganiami uczelni 
% UWAGA! - wydruk jest dwustronny i jest to zabieg zamierzony

%%%%%%%%%%%%%%%%%%%%%%%%%%%%%%%%%%%%%%%%%
% Plik konfigurujący
% Szablon pracy dyplomowej
% Wydział Informatyki 
% Zachodniopomorski Uniwersytet Technologiczny w Szczecinie
% autor Joanna Kołodziejczyk (jkolodziejczyk@zut.edu.pl)
% Bardzo wczesnym pierwowzorem szablonu był
% The Legrand Orange Book
% Version 2.1 (26/09/2018)
%
% Modifications to LOB assigned by %JK
%%%%%%%%%%%%%%%%%%%%%%%%%%%%%%%%%%%%%%%%%


%----------------------------------------------------------------------------------------
%	VARIOUS REQUIRED PACKAGES AND CONFIGURATIONS
%----------------------------------------------------------------------------------------

%GEOMETRY
\usepackage[top=3.2cm,bottom=3.5cm,left=2.5cm,right=2.5cm,headsep=1.5ex,bindingoffset=1cm,a4paper]{geometry} % Page margins

% GRAPHICS
\usepackage{graphicx} % Required for including pictures
\graphicspath{{Pictures/}} % Specifies the directory where pictures are stored

% For math equations, theorems, symbols, etc
\usepackage{amsmath,amsfonts,amssymb,amsthm} 

% Customize lists
\usepackage{enumitem} 
\setlist{nolistsep} % Reduce spacing between bullet points and numbered lists
\setlist[itemize]{label=--}

% Required for nicer horizontal rules in tables
\usepackage{booktabs} 

\usepackage{xcolor} % Required for specifying colors by name
% Define blue colors used for highlighting throughout the book- based on the WI ZUT colors
\definecolor{blueWI}{cmyk}{.6,.2,0,.0} % JK - Define the blue colour used for highlighting throughout the book
\definecolor{blueZUT}{cmyk}{1,.75,0,.2} % JK - Define the blue colour used for highlighting throughout the book
\definecolor{grayZUT}{cmyk}{0,0,0,0.4} % JK - Define the blue colour used for highlighting throughout the book

%other definision
\newcommand{\rulecolor}[1]{\color{#1}\rule}

%----------------------------------------------------------------------------------------
%	FONTS AND LANGUAGE (JK - configuring and styling)
%----------------------------------------------------------------------------------------
%
\usepackage{newtxmath,newtxtext}
\usepackage{t1enc}
\usepackage[polish]{babel}% moved after to avoid conflict between polish babel and amsmath
\usepackage[utf8]{inputenc} 
%\usepackage{avant} % Use the Avantgarde font for headings

%----------------------------------------------------------------------------------------
%	PAGE HEADERS AND FOOTERS (JK - Styling for the current chapter in the header)
%----------------------------------------------------------------------------------------

\usepackage{fancyhdr} % Required for header and footer configuration
\setlength{\headheight}{2.5ex}
\pagestyle{fancy} % Enable the custom headers and footers

\renewcommand{\chaptermark}[1]{\markboth{\sffamily\normalsize\thechapter.\hspace{5pt} #1}{}} % JK - Styling for the current chapter in the header
\renewcommand{\sectionmark}[1]{\markright{\sffamily\normalsize\thesection\hspace{5pt} #1}{}} % Styling for the current section in the header

\fancyhf{} % Clear default headers and footers

% JK - header with page hanging  and chapter title in the box
\fancyhead[EL]{%
  \textcolor{white}{%
    \llap{%
      \colorbox{blueZUT}{%
        \makebox[7ex][r]{\sffamily\thepage}%
      }%
      \hspace{1.25\marginparsep}%
      \hspace{-\fboxsep}%
    }%
    \textcolor{black}{%
      \colorbox{blueWI!20}{%
        \makebox[\textwidth][l]{\sffamily\shorttitle}}}
     \addtolength{\headheight}{5ex} % Increase the spacing around the header slightly
  }%
}
\fancyhead[OR]{%
  \textcolor{white}{%
    \llap{%
      \colorbox{blueZUT}{%
        \makebox[7ex][r]{\sffamily\thepage}%
      }%
      \hspace{1.25\marginparsep}%
      \hspace{-\fboxsep}%
    }%
    \textcolor{black}{%
      \colorbox{blueWI!20}{%
        \makebox[\textwidth][l]{\sffamily\leftmark}}}%{\rightmark}}}
    \addtolength{\headheight}{5ex} % Increase the spacing around the header slightly    
  }%
}

\fancypagestyle{plain}{%
   \fancyhead{} % get rid of headers
   \fancyfoot[RE,RO]{
  	\textcolor{white}{%
    	\llap{%
      	\colorbox{blueZUT}{%
        	\makebox[7ex][l]{\sffamily\thepage}%
      }%
      \hspace{-6\marginparsep}%separation from the margin
      \hspace{-5\fboxsep}%
     }%
     }%
    } 
   \addtolength{\headheight}{18pt} % Increase the spacing around the header slightly
}


\renewcommand*{\headrulewidth}{0pt}
\renewcommand*{\footrulewidth}{0pt}

% Removes the header from odd empty pages at the end of chapters
\makeatletter
\renewcommand{\cleardoublepage}{
\clearpage\ifodd\c@page\else
\hbox{}
\vspace*{\fill}
\thispagestyle{empty}
\newpage
\fi}

%----------------------------------------------------------------------------------------
%	BIBLIOGRAPHY (JK - configuring and styling)
%----------------------------------------------------------------------------------------
% 
\usepackage[
style=numeric,% style alphabetic or numeric
citestyle=verbose,
sorting=nyt,%name -year -title
sortcites=true,
autopunct=true,
autolang=hyphen,
hyperref=true, % if the citation is the link to bibliography
backend=biber,
defernumbers=true]{biblatex}
\addbibresource{bibliography.bib} % BibTeX bibliography file
\defbibheading{bibempty}{}
\nocite{*}

%\usepackage{calc} % For simpler calculation - used for spacing the index letter headings correctly

%----------------------------------------------------------------------------------------
%	CHAPTER & SECTION HEADINGS
%----------------------------------------------------------------------------------------

\usepackage[explicit]{titlesec}

% \titleformat{<command>}[<shape>]{<format>}{<label>}{<sep>}{<before-code>}[<after-code>]

\titleformat{\chapter}[block]
  {\huge\sffamily\color{blueZUT}}
  {\hspace{- 3ex}{\thechapter.} \hspace{0.5em}{#1\strut}}
  {0pt}
  {}

\titleformat{name = \chapter, numberless}[block]
  {\huge\sffamily\color{blueZUT}}
  {{#1\strut}}
  {0pt}
  {}

%----------------------------------------------------------------------------------------
%	Hanging SECTION NUMBERING (MARGIN)
%----------------------------------------------------------------------------------------

\makeatletter
\renewcommand{\@seccntformat}[1]{\llap{\textcolor{blueZUT}{\csname the#1\endcsname}\hspace{1em}}}                    

\renewcommand{\section}
{\@startsection{section}{1}{\z@}
{-4ex \@plus -1ex \@minus -.4ex}
{1ex \@plus.2ex }
{\normalfont\large\sffamily\bfseries}}

\renewcommand{\subsection}{\@startsection {subsection}{2}{\z@}
{-3ex \@plus -0.1ex \@minus -.4ex}
{0.5ex \@plus.2ex }
{\normalfont\sffamily\bfseries}}

\renewcommand{\subsubsection}{\@startsection {subsubsection}{3}{\z@}
{-2ex \@plus -0.1ex \@minus -.2ex}
{.2ex \@plus.2ex }
{\normalfont\small\sffamily\bfseries}}                        

\renewcommand\paragraph{\@startsection{paragraph}{4}{\z@}
{-2ex \@plus-.2ex \@minus .2ex}
{.1ex}
{\normalfont\small\sffamily\bfseries}}



%----------------------------------------------------------------------------------------
%	MAIN TABLE OF CONTENTS (JK modification: style, indentation, colors,)
%----------------------------------------------------------------------------------------

\usepackage{titletoc} % Required for manipulating the table of contents
\contentsmargin{0cm} % Removes the default margin

% Chapter text styling
\titlecontents{chapter}[1.25cm] % Indentation
{\addvspace{12pt}\large\sffamily} % Spacing and font options for chapters
{\color{blueZUT!60}\contentslabel[\Large\thecontentslabel]{1.25cm}\color{blueZUT}} % JK Chapter number
{\color{blueZUT}}  
{\color{blueZUT!60}\normalsize\;\titlerule*[.5pc]{.}\;\thecontentspage} % Page number

% Section text styling
\titlecontents{section}
	[1.25cm] % Left indentation
	{\addvspace{3pt}\sffamily} % Spacing and font options for sections
	{\contentslabel[\thecontentslabel]{1.25cm}} % Formatting of numbered sections of this type
	{} % Formatting of numberless sections of this type
	{\;\titlerule*[.5pc]{.}\;\color{black}\thecontentspage} % Formatting of the filler to the right of the heading and the page number

% Subsection text styling
\titlecontents{subsection}
	[2.5cm] % Left indentation
	{\addvspace{1pt}\sffamily\small} % Spacing and font options for subsections
	{\contentslabel[\thecontentslabel]{1.25cm}} % Formatting of numbered sections of this type
	{} % Formatting of numberless sections of this type
	{\ \titlerule*[.5pc]{.}\;\thecontentspage} % Formatting of the filler to the right of the heading and the page number

%%%%% JK Add figures and tables
% Figure text styling
\titlecontents{figure}
	[0em] % Left indentation
	{\addvspace{3pt}\sffamily} % Spacing and font options for figures
	{\contentslabel[\thecontentslabel]{1.25cm}} % Formatting of numbered sections of this type
	{} % Formatting of numberless sections of this type
	{\ \titlerule*[.5pc]{.}\;\thecontentspage} % Formatting of the filler to the right of the heading and the page number

% Table text styling
\titlecontents{table}
	[0em] % Left indentation
	{\addvspace{3pt}\sffamily} % Spacing and font options for tables
	{\contentslabel[\thecontentslabel]{1.25cm}} % Formatting of numbered sections of this type
	{} % Formatting of numberless sections of this type
	{\ \titlerule*[.5pc]{.}\;\thecontentspage} % Formatting of the filler to the right of the heading and the page number


%----------------------------------------------------------------------------------------
%	THEOREM STYLES
%----------------------------------------------------------------------------------------

\newcommand{\intoo}[2]{\mathopen{]}#1\,;#2\mathclose{[}}
\newcommand{\ud}{\mathop{\mathrm{{}d}}\mathopen{}}
\newcommand{\intff}[2]{\mathopen{[}#1\,;#2\mathclose{]}}
\renewcommand{\qedsymbol}{$\blacksquare$}
\renewcommand{\thmname}{Twierdzenie}

% Boxed/framed environments
\newtheoremstyle{blueZUTnumbox}% Theorem style name
{0pt}% Space above
{0pt}% Space below
{\normalfont}% Body font
{}% Indent amount
{\small\bf\sffamily\color{blueZUT}}% Theorem head font
{\;}% Punctuation after theorem head
{0.25em}% Space after theorem head
{\small\sffamily\color{blueZUT}\thmname{#1}\nobreakspace\thmnumber{\@ifnotempty{#1}{}\@upn{#2}}% Theorem text (e.g. Theorem 2.1)
\thmnote{\nobreakspace\the\thm@notefont\sffamily\bfseries\color{black}---\nobreakspace#3.}} % Optional theorem note

% \newtheoremstyle{blacknumex}% Theorem style name
% {5pt}% Space above
% {5pt}% Space below
% {\normalfont}% Body font
% {} % Indent amount
% {\small\bf\sffamily}% Theorem head font
% {\;}% Punctuation after theorem head
% {0.25em}% Space after theorem head
% {\small\sffamily{\tiny\ensuremath{\blacksquare}}\nobreakspace\thmname{#1}\nobreakspace\thmnumber{\@ifnotempty{#1}{}\@upn{#2}}% Theorem text (e.g. Theorem 2.1)
% \thmnote{\nobreakspace\the\thm@notefont\sffamily\bfseries---\nobreakspace#3.}}% Optional theorem note

\newtheoremstyle{blacknumbox} % Theorem style name
{5pt}% Space above
{5pt}% Space below
{\normalfont}% Body font
{}% Indent amount
{\small\bf\sffamily}% Theorem head font
{\;}% Punctuation after theorem head
{0.25em}% Space after theorem head
{\small\sffamily\thmname{#1}\nobreakspace\thmnumber{\@ifnotempty{#1}{}\@upn{#2}}% Theorem text (e.g. Theorem 2.1)
\thmnote{\nobreakspace\the\thm@notefont\sffamily\bfseries---\nobreakspace#3.}}% Optional theorem note

% Non-boxed/non-framed environments
\newtheoremstyle{blueZUTnum}% Theorem style name
{5pt}% Space above
{5pt}% Space below
{\normalfont}% Body font
{}% Indent amount
{\small\bf\sffamily\color{blueZUT}}% Theorem head font
{\;}% Punctuation after theorem head
{0.25em}% Space after theorem head
{\small\sffamily\color{blueZUT}\thmname{#1}\nobreakspace\thmnumber{\@ifnotempty{#1}{}\@upn{#2}}% Theorem text (e.g. Theorem 2.1)
\thmnote{\nobreakspace\the\thm@notefont\sffamily\bfseries\color{black}---\nobreakspace#3.}} % Optional theorem note
\makeatother

% Defines the theorem text style for each type of theorem to one of the three styles above
\newcounter{dummy} 
\numberwithin{dummy}{section}
\theoremstyle{blueZUTnumbox}
\newtheorem{theoremeT}[dummy]{Twierdzenie}
\theoremstyle{blueZUTnum}
\newtheorem{exampleT}{Przykład}[chapter]
\theoremstyle{blacknumbox}
\newtheorem{definitionT}{Definicja}[section]


%----------------------------------------------------------------------------------------
%	DEFINITION OF COLORED BOXES
%----------------------------------------------------------------------------------------

\RequirePackage[framemethod=default]{mdframed} % Required for creating the theorem, definition, exercise and corollary boxes

% Theorem box
\newmdenv[skipabove=7pt,
skipbelow=7pt,
backgroundcolor=black!3,
linecolor=blueZUT,
innerleftmargin=5pt,
innerrightmargin=5pt,
innertopmargin=5pt,
leftmargin=0cm,
rightmargin=0cm,
innerbottommargin=5pt]{tBox}


% Definition box
\newmdenv[skipabove=7pt,
skipbelow=7pt,
rightline=false,
leftline=true,
topline=false,
bottomline=false,
linecolor=blueZUT,
innerleftmargin=5pt,
innerrightmargin=5pt,
innertopmargin=0pt,
leftmargin=0cm,
rightmargin=0cm,
linewidth=2pt,
innerbottommargin=0pt]{dBox}	

% Creates an environment for each type of theorem and assigns it a theorem text style from the "Theorem Styles" section above and a colored box from above
\newenvironment{theorem}{\begin{tBox}\begin{theoremeT}}{\end{theoremeT}\end{tBox}}				  
\newenvironment{definition}{\begin{dBox}\begin{definitionT}}{\end{definitionT}\end{dBox}}	
\newenvironment{example}{\begin{exampleT}}{\hfill{\tiny\ensuremath{\blacksquare}}\end{exampleT}}		



%----------------------------------------------------------------------------------------
%	LISTING ENVIRONMENT
%----------------------------------------------------------------------------------------

\usepackage{listings}

%Polish set of letteres accepted in the listings
\lstset{
literate=%
{ą}{{\k{a}}}1
{Ą}{{\k{A}}}1
{ć}{{\'c}}1
{Ć}{{\'{C}}}1
{ę}{{\k{e}}}1
{Ę}{{\k{E}}}1
{ł}{{\l{}}}1
{Ł}{{\L{}}}1
{ń}{{\'n}}1
{Ń}{{\'N}}1
{ó}{{\'o}}1
{Ó}{{\'O}}1
{ś}{{\'s}}1
{Ś}{{\'S}}1
{ż}{{\.z}}1
{Ż}{{\.Z}}1
{ź}{{\'z}}1
{Ź}{{\'Z}}1
}

\renewcommand{\lstlistingname}{\small\sffamily\bfseries\color{blueZUT} Algorytm} % Change default listing caption to Algorthm
\renewcommand{\lstlistlistingname}{Lista \lstlistingname ów}

\definecolor{codegreen}{rgb}{0,0.6,0}
\definecolor{codegray}{rgb}{0.5,0.5,0.5}

\lstdefinestyle{mystyle}{
 %   backgroundcolor=\color{grayZUT!10},   
    basicstyle= \scriptsize\fontfamily{lmss}\selectfont,%\footnotesize\fontfamily{cmss}\selectfont,
    commentstyle=\color{codegray},
    keywordstyle=\color{violet},
    numberstyle=\tiny\color{codegray},%numeracja linijek
    identifierstyle={\color{black}},
    numbers=left,%numeracja linijek
    numbersep=10pt,%numeracja linijek
    stringstyle=\color{codegreen},
    breakatwhitespace=true,         
    breaklines=true,                 
    captionpos=b,                    
    %keepspaces=false,                
    %showspaces=false,                
    showstringspaces=false,
    showtabs=true,                  
    tabsize=2,
    frame=leftline,
    rulecolor = \color{blueWI},
    xleftmargin=5ex,
    xrightmargin=5ex
    }
 
\lstset{style=mystyle}

%----------------------------------------------------------------------------------------
% CAPTIONS ( JK - design and implementation)
%----------------------------------------------------------------------------------------

\usepackage{caption}
\captionsetup[figure]{name={\small\sffamily\color{blueZUT} Rysunek}}
\captionsetup[table]{name={\small\sffamily\color{blueZUT} Tabela}}
\captionsetup{font={small,sf,singlespacing}}


%----------------------------------------------------------------------------------------
%	HYPERLINKS IN THE DOCUMENTS
%----------------------------------------------------------------------------------------

\usepackage{hyperref}
%\hypersetup{hidelinks,backref=true,pagebackref=true,hyperindex=true,colorlinks=false,breaklinks=true,urlcolor=blueZUT,bookmarks=true,bookmarksopen=false}
\hypersetup{hidelinks,breaklinks=true,urlcolor=blueZUT,bookmarksopen=false,pdftitle={Title},pdfauthor={Author}}

%----------------------------------------------------------------------------------------
%	Listings languages
%----------------------------------------------------------------------------------------


\lstdefinelanguage{JavaScript}{
  keywords={typeof, new, true, false, catch, function, return, null, catch, switch, var, if, in, while, do, else, case, break},
  keywordstyle=\color{blue}\bfseries,
  ndkeywords={class, export, boolean, throw, implements, import, this},
  ndkeywordstyle=\color{darkgray}\bfseries,
  identifierstyle=\color{black},
  sensitive=false,
  comment=[l]{//},
  morecomment=[s]{/*}{*/},
  commentstyle=\color{purple}\ttfamily,
  stringstyle=\color{red}\ttfamily,
  morestring=[b]',
  morestring=[b]"
}

\lstdefinelanguage{CSharp}
{
 morecomment = [l]{//}, 
 morecomment = [l]{///},
 morecomment = [s]{/*}{*/},
 morestring=[b]", 
 sensitive = true,
 morekeywords = {abstract,  event,  new,  struct,
   as,  explicit,  null,  switch,
   base,  extern,  object,  this,
   bool,  false,  operator,  throw,
   break,  finally,  out,  true,
   byte,  fixed,  override,  try,
   case,  float,  params,  typeof,
   catch,  for,  private,  uint,
   char,  foreach,  protected,  ulong,
   checked,  goto,  public,  unchecked,
   class,  if,  readonly,  unsafe,
   const,  implicit,  ref,  ushort,
   continue,  in,  return,  using,
   decimal,  int,  sbyte,  virtual,
   default,  interface,  sealed,  volatile,
   delegate,  internal,  short,  void,
   do,  is,  sizeof,  while,
   double,  lock,  stackalloc,   
   else,  long,  static,   
   enum,  namespace,  string}
}

\lstdefinelanguage{IL}{%
  % so listings can detect directives and register names
  alsoletter={.\$},
  % strings, characters, and comments
  morestring=[b]",
  morestring=[b]',
  morecomment = [l]{//}, 
  % instructions
  morekeywords={[1]ldstr, ldstr, call, callvirt, instance, ldloc, ldc.i4,
   stloc, conv.r4, ldc.r4, ldloca.s, mul, add, div, sub, ldc.i4.0, cgt, ceq,
   br, brfalse},
  % assembler directives
  morekeywords={[2]string, void, int32, float32, bool},
}[strings,comments,keywords]
 % JK - Plik zawierający podstawowe elementy konfigurujące układ dokumentu
% UWAGA! - raczej nie będzie potrzeby zmieniania jego struktury

%%%%%%%%%%%%%%%%%%%%%%%%%%%%%%%%%%%%%%%%%
% Plik z definicjami
% Szablon pracy dyplomowej
% Wydział Informatyki 
% Zachodniopomorski Uniwersytet Technologiczny w Szczecinie
% autor Joanna Kołodziejczyk (jkolodziejczyk@zut.edu.pl)
% Bardzo wczesnym pierwowzorem szablonu był
% The Legrand Orange Book
% Version 2.1 (26/09/2018)
%
% Modifications to LOB assigned by %JK
%%%%%%%%%%%%%%%%%%%%%%%%%%%%%%%%%%%%%%%%%



\def\HRule{\color{blueWI} \rule{\linewidth}{0.6pt}}

%----------------------------------------------------------------------------------------
% Typ pracy (wybrać właściwy)
%----------------------------------------------------------------------------------------
\def\degreename{praca dyplomowa magisterska} 
%\def\degreename{praca dyplomowa inżynierska}

%----------------------------------------------------------------------------------------
% Temat pracy
%----------------------------------------------------------------------------------------
%\def\ttitle{Szablon pracy dyplomowej inżynierskiej lub magisterskiej, do wykorzystania przez studentów Wydziału Informatyki Zachodniopomorskiego Uniwersytetu Technologicznego}
\def\ttitle{Projekt i implementacja kompilatora języka JavaScript na platformę .NET.}
\def\shorttitle{Projekt i implementacja kompilatora języka JavaScript na platformę .NET.} %Jeżeli tytuł pracy jest na tyle długi, że zajmuje 3 linijki to trzeba podać krótszy ekwiwalent do nagłówków stron parzystych
\def\ttitleEng{Design and implementation of the JavaScript compiler into the .NET platform.} %temat pracy w j. angielskim


%----------------------------------------------------------------------------------------
% Informacje o autorze
%----------------------------------------------------------------------------------------
\def\authornames{Przemysław Gawlas} %imię i nazwisko autora
\def\albumno{gp36035} %numer albumu
\def\speciality{Inżynieria oprogramowania} %nazwa specjalności
\def\field{Informatyka} %dziedzina nauki
\def\studyform{studia niestacjonarne} %forma studiów

%----------------------------------------------------------------------------------------
% Informacje o promotorze
%----------------------------------------------------------------------------------------
\def\supname{dr~inż.~Piotr~Błaszyński} %imię i nazwisko promotora
\def\departmentname{Katedra Inżynierii Oprogramowania i Cyberbezpieczeństwa} %nazwa katedry promotora

%----------------------------------------------------------------------------------------
% Data wydania tematu pracy
%----------------------------------------------------------------------------------------
\def\datetitle{05.10.2020}

%----------------------------------------------------------------------------------------
% Rok i miejsce złożenia pracy
%----------------------------------------------------------------------------------------
\def\placesubmit{Szczecin}
\def\yearsubmit{2021}
  % JK - dodatkowe definicje głównie treść strony tytułowej
% UWAGA! - konieczność edycji celem zmiany autora/tematu/dat itp., itd

%----------------------------------------------------------------------------------------
% OTWARCIE DOKUMENTU
%----------------------------------------------------------------------------------------
\begin{document}

%----------------------------------------------------------------------------------------
% STRONA TYTUŁOWA 
%----------------------------------------------------------------------------------------
\title
 {Projekt i implementacja kompilatora języka JavaScript na platformę .NET.}
\author[Przemysław Gawlas]
	{Przemysław Gawlas\\[3mm]Opiekun pracy: dr inż. Piotr Błaszyński}

\addtobeamertemplate{navigation symbols}{}{
    \usebeamerfont{footline}
    \usebeamercolor[fg]{footline}
    \hspace{1em}
    \insertframenumber/\inserttotalframenumber
}

\date{}

\begin{frame}
	\titlepage
\end{frame}


%----------------------------------------------------------------------------------------
% PUSTA STRONA PO STRONIE TYUŁOWEJ (JK - design and implementation)
%----------------------------------------------------------------------------------------
\newpage
\thispagestyle{empty}
~\vfill

%----------------------------------------------------------------------------------------
% STRONA Z OŚWIADCZENIEM AUTORA (JK - design and implementation)
%----------------------------------------------------------------------------------------
\newpage
\thispagestyle{empty}
%%%%%%%%%%%%%%%%%%%%%%%%%%%%%%%%%%%%%%%%%
% Strona z oświadczeniem wymaganym przez ZUT
% Szablon pracy dyplomowej
% Wydział Informatyki 
% Zachodniopomorski Uniwersytet Technologiczny w Szczecinie
% autor Joanna Kołodziejczyk (jkolodziejczyk@zut.edu.pl)
% Bardzo wczesnym pierwowzorem szablonu był
% The Legrand Orange Book
% Version 2.1 (26/09/2018)
%
% Modifications to LOB assigned by %JK
%%%%%%%%%%%%%%%%%%%%%%%%%%%%%%%%%%%%%%%%%


\begin{center}
\noindent 
{{\color{blueZUT}\Large\sffamily {Oświadczenie\\[.2cm]
autora pracy dyplomowej}}}\\[1cm] 
\end{center}

\noindent Oświadczam, że \degreename {\ }pn.{\ }{\emph \ttitle}{\ }
napisana pod kierunkiem \supname  {\ }
jest w całości moim samodzielnym autorskim opracowaniem sporządzonym przy wykorzystaniu wykazanej w pracy literatury przedmiotu i materiałów źródłowych. 
Złożona w dziekanacie Wydziału Informatyki
treść  mojej pracy dyplomowej w formie elektronicznej jest zgodna z treścią w formie pisemnej.

\vspace{0.2cm}

\noindent Oświadczam ponadto, że złożona w dziekanacie praca dyplomowa ani jej fragmenty nie były wcześniej przedmiotem procedur procesu dyplomowania związanych z uzyskaniem tytułu zawodowego w uczelniach wyższych.

\vspace{3.5cm}

\noindent {Podpis autora:}\dotfill % Printing/edition date

\vspace{1.5cm}

\noindent {Szczecin, dnia:\dotfill} % Printing/edition date

%----------------------------------------------------------------------------------------
%	STRESZCZENIE I SŁOWA KLUCZOWE (1 STRONA) (JK - design and implementation)
%----------------------------------------------------------------------------------------
\newpage
\thispagestyle{empty}
%%%%%%%%%%%%%%%%%%%%%%%%%%%%%%%%%%%%%%%%%
% Specjalna strona pracy ze streszczeniem i abstractem w j. angielskim
% Szablon pracy dyplomowej
% Wydział Informatyki 
% Zachodniopomorski Uniwersytet Technologiczny w Szczecinie
% autor Joanna Kołodziejczyk (jkolodziejczyk@zut.edu.pl)
% Bardzo wczesnym pierwowzorem szablonu był
% The Legrand Orange Book
% Version 2.1 (26/09/2018)
%
% Modifications to LOB assigned by %JK
%%%%%%%%%%%%%%%%%%%%%%%%%%%%%%%%%%%%%%%%%


\begin{center}
\noindent {{\color{blueZUT}\Large\sffamily  {Streszczenie}}}\\[1cm] 
\end{center}
Celem niniejszej pracy było zaprojektowanie oraz implementacja kompilatora języka JavaScript platformę .NET. Praca przedstawia dostępne narzędzia, technologie oraz języki programowania wykorzystane w projekcie oraz technologie pokrewne które mogły by być wykorzystane. Przedstawia częściową analizę języka JavaScript na podstawie której wyznaczany jest zakres implementacji kompilatora. Praca przedstawia projekt oraz wyjaśnia implementację kompilatora. W celu weryfikacji poprawności działania, zostały przygotowane programy testujące.

\vspace{10pt}
\noindent{\bf słowa kluczowe:} kompilator, JavaScript, .NET

\vfill

\begin{center}
\noindent {{\color{blueZUT}\Large\sffamily {Abstract}}}\\[1cm] 
\end{center}
The aim of this diploma thesis was to design and implement a JavaScript language compiler for .NET platform. The work presents the available tools, technologies and programming languages used in the project and related technologies that could be used. It presents a partial analysis of the JavaScript language on the basis of which the scope of the compiler implementation is determined. The work presents the project and explains the implementation of the compiler. In order to verify the correctness of operation, testing programs have been prepared. 

\vspace{10pt}
\noindent{\bf keywords:} compiler, JavaScript, .NET
 %abstract.tex 
 
%----------------------------------------------------------------------------------------
%	SPIS TREŚCI (JK - design and implementation)
%----------------------------------------------------------------------------------------
\pagestyle{empty}  % Wyłącz stopkę i nagłówek w TOC
\tableofcontents % wyświetl spis
\addtocontents{toc}{\protect\thispagestyle{empty}} % Zachowaj pusty nagłówek i stopkę w TOC
% Wymuszenie rozpoczęcie pierwszego rozdziału na nieparzystej stronie, aby znajdował się po prawej stronie 
\cleardoublepage 
% Ponownie włącz nagłówki i stopki
\pagestyle{fancy} 

%----------------------------------------------------------------------------------------
%	WSTĘP W PRACY DYPLOMOWEJ (JK - design and implementation)
%----------------------------------------------------------------------------------------
\addcontentsline{toc}{chapter}{Wstęp}
%%%%%%%%%%%%%%%%%%%%%%%%%%%%%%%%%%%%%%%%%
% Plik z wstępem do pracy
% Szablon pracy dyplomowej
% Wydział Informatyki 
% Zachodniopomorski Uniwersytet Technologiczny w Szczecinie
% autor Joanna Kołodziejczyk (jkolodziejczyk@zut.edu.pl)
% Bardzo wczesnym pierwowzorem szablonu był
% The Legrand Orange Book
% Version 2.1 (26/09/2018)
%
% Modifications to LOB assigned by %JK
%%%%%%%%%%%%%%%%%%%%%%%%%%%%%%%%%%%%%%%%%


\chapter*{Wstęp}

\par W branży programistycznej istnieje bardzo dużo języków programowania oraz różnych środowisk uruchomieniowych dla nich przeznaczonych. Podczas tworzenia oprogramowania niezbędny jest dla programisty kompilator. Zamienia on kod programu napisanego w konkretnym języku programowania, na zrozumiały dla środowiska uruchomieniowego ciąg instrukcji. Warto podkreślić, że języki programowania, które są proste do zrozumienia i opanowania dla programistów oraz pozwalają na pisanie programów, które można uruchomić na różnych urządzeniach i umożliwiają tworzenie bardzo zróżnicowanych rodzajów oprogramowania, są o wiele częściej używane niż inne. Powoduje to, że powstaje dużo różnych bibliotek i frameworków ułatwiających i automatyzujących tworzenie oprogramowania.

\par Jednym z popularnych języków wśród programistów jest język JavaScript, który może być uruchomiany w przeglądarkach internetowych lub w maszynie wirtualnej takiej jak Node.js. Innym z popularnych środowisk uruchomieniowych jest platforma .NET, dla której istnieje wiele kompilatorów różnych języków. Wykorzystanie istniejących bibliotek, frameworków czy modułów napisanych w języku JavaScript w niezmienionej formie na platformie .NET, umożliwi programistom na tworzenie bardziej uniwersalnego kodu.

\par \textbf{Celem} niniejszej pracy jest \textbf{zaprojektowanie} oraz \textbf{implementacja kompilatora} języka \textbf{JavaScript} na kod \textbf{IL Assembler} uruchamiany na \textbf{platformie .NET}. Następnie w celu sprawdzenia poprawności działania kompilatora, zostaną przeprowadzone testy przy pomocy prostych implementacji kodu JavaScript. Zostaną również zaprojektowane oraz zaimplementowane dwa bardziej skomplikowane testy, do \textbf{porównania maszyn wirtualnych Node.js oraz .NET}. Zostanie także porównany kod assemblera generowanego przez kompilator .Net Framework języka C\# z kodem generowanym przez implementowany kompilator.

\par Praca podzielona jest na cztery części. Pierwsza część opisuje pojęcia i technologie wykorzystywane do realizacji tego projektu oraz technologie pokrewne, które w pewnym stopniu realizują cel pracy lub realizują podobne założenia. Druga część poświęcona jest zaplanowaniu zakresu implementacji tworzonego kompilatora oraz analizie języka JavaScript. Kolejna część opisuje sposób implementacji tego kompilatora na podstawie zdefiniowanych założeń. Ostatnia część opisuje przeprowadzone testy realizowane w ramach pracy oraz przedstawia wyniki ich działania.
 % introduction.tex zawiera treść wstępu

%----------------------------------------------------------------------------------------
%	ROZDZIAŁ 1
%----------------------------------------------------------------------------------------
\chapter{Pojęcia i technologie}
\label{rozdzial1}

\section{Zakres projektu}
\index{Zakres projektu}

Projekt będzie realizował zaprojektowanie i implementację kompilatora języka JavaScript na platformę uruchomieniową \textit{.NET}. W tym rozdziale zostaną opisane pojęcia oraz technologie, które będą wykorzystywane w realizacji tego projektu. Na początku opisane będą takie pojęcia jak \textit{kompilator} oraz \textit{maszyna wirtualna}. Następie zostaną omówione technologie wiodące w projekcie takie jak \textit{JavaScript}, \textit{Node.js}, \textit{.NET Framework} oraz \textit{IL Assembler}.

\subsection{Kompilator}
\index{Kompilator}
W dzisiejszych czasach niezbędnym narzędziem programisty jest kompilator. Jest to narzędzie, którego zadaniem jest tłumaczenie programu napisanego przez programistę, na program który będzie można uruchomić na konkretnym środowisku uruchomieniowym.
\par Mówiąc ściślej kompilator to program napisany w języku implementacyjnym, który odczytuje język źródłowy i tłumaczy go na język wynikowy. Proces zamiany kodu źródłowego na wynikowy nazywany jest \textbf{kompilacją}. Kodem wynikowym procesu kompilacji może być od razu kod maszynowy, który interpretowany jest bezpośrednio przez procesor lub maszynę wirtualną, albo do kodu pośredniego, który też może zostać skompilowany przez inny kompilator.
\par Kompilatory mogą być napisane w dowolnym języku programowania. Istnieje kilka specjalnie zaprojektowanych do tego zadania języków, takie jak \textit{Pascal} czy \textit{Algol 68}. Nie mniej jednak, wybór języka do implementacji kompilatora przez twórcę, powinien się opierać na założeniu, że powinien on zminimalizować wysiłek implementacyjny i zmaksymalizować jakość kompilatora.
\par Język źródłowy który przetwarzany jest przez kompilator prawie zawsze jest oparty na wcześniej zdefiniowanej gramatyce. Dzięki temu program kompilatora potrafi odróżnić od siebie kolejne instrukcje i zamienić je na równoważne ciągi instrukcji w języku docelowym.
\footcite[1-4]{Mckeeman1974}
\par Podobnym w działaniu jest \textbf{interpreter}, który tak jak kompilator, jest pisany w jednym implementacyjnym oraz odczytuje język kodu źródłowego, ale nie produkuje kodu wynikowego, tylko odczytany kod jest od razu wykonywany. Niektóre języki przyjmują schematy zawierające wykorzystanie kompilatora oraz interpretera w procesie wytwarzania oprogramowania. Jednym z przykładów jest język \textbf{Java}, który kompilowany jest do postaci nazywanej \textit{bytecode}, a następnie interpretowany jest przez maszynę wirtualną Java (Java Virtual Machine, JVM).\footcite[3,4]{EngineeringCompiler}

\subsection{Maszyna wirtualna}
% https://ieeexplore.ieee.org/stamp/stamp.jsp?tp=&arnumber=1430629

\par Wirtualizacja stała się ważnym narzędziem w projektowaniu systemów komputerowych, a maszyny wirtualne używane są w wielu obszarach informatycznych, od systemów operacyjnych po architektury procesorów języków programowania.
\par Dla programistów oraz użytkowników wirtualizacja likwiduje tradycyjne interfejsy oraz ograniczenia zasobów związanych z różnymi urządzeniami. Maszyny wirtualne zwiększają interoperacyjność oprogramowania oraz wszechstronność platformy, dlatego też często się je wykorzystuje.
\par Maszyna wirtualna to nic innego jak program uruchamiany na prawdziwej maszynie, który potrafi obsługiwać pożądaną architekturę. W ten sposób można obejść rzeczywistą kompatybilność maszyny i ograniczenia zasobów sprzętowych. Pozwala to, między innymi na równoczesne tworzenie oprogramowania dla wielu platform, bez konieczności stosowania bezpośrednio interfejsów rzeczywistej maszyny, a jedynie wykorzystanie tych udostępnianych przez maszynę wirtualną. \footcite{Smith2005}

\subsection{JavaScript}
Jest to skryptowy język programowania, dzięki którego można realizować aplikacje w paradygmacie imperatywnym, obiektowym oraz funkcyjnym. Najczęściej jest wykorzystywany w stronach internetowych, gdzie kolejne instrukcje wykonywane są przez przeglądarkę sieci Web, ale również zyskuje popularność w innych środowiskach. \footcite{aboutJS}
\par JavaScript został wdrożony w roku 1995 roku, jako sposób dodawania programów do stron internetowych. Jako pierwszą przeglądarką obsługującą JavaScript to Netscape Navigator. Następnie inne, głównie graficzne przeglądarki wprowadzały możliwość uruchamiania kodu napisanego w JavaScript. Umożliwiło to tworzenie nowoczesnych stron internetowych z którymi można było bezpośrednio współpracować, bez konieczności ponownego pobierania strony po każdej wykonanej akcji.
\par W momencie kiedy zaczęto używać JavaScript poza Netscape, został stworzony dokument standaryzujący, który opisuje sposób działania języka. Utworzono go, aby wszystkie nowo tworzone oprogramowanie mające wykorzystywać JavaScript, faktycznie używały tego samego języka. Dokument ten nazywany jest standardem \textbf{ECMAScript}, który został nazwany po organizacji Ecma International, twórców tego dokumentu.
\par JavaScript jest językiem bardzo elastycznym, przez co ma też swoje wady i zalety. Przez swoją elastyczność pozwala na wykorzystywanie wielu technik i praktyk programistycznych które mogą być niemożliwe w innych językach.
\par Jako, że jest to język skryptowy, to tak jak podobne tego typu języki posiada dynamiczne typowanie zmiennych. Oznacza to, że każda ze zmiennej jest definiowana poprzez słowo kluczowe \texttt{var}, a w nowszej wersji można to zrobić już przy pomocy dwóch różnych słów \texttt{const} oraz \texttt{let}. Kolejną z podstawowych rzeczy w JavaScript są funkcje. Dzięki nim można pisać programy we wspomnianych wcześniej paradygmatach. Pozwalają one nie tylko na wydzielenie kodu na mniejsze części ale również na definiowanie bardziej złożonych struktur czy klas.\footcite{EloquentJavaScript}
% https://eloquentjavascript.net/Eloquent_JavaScript.pdf

\subsection{Node.js}
Jest to asynchroniczne środowisko uruchomieniowe dla języka JavaScript. Node.js został zaprojektowany do tworzenia skalowalnych aplikacji sieciowych. Pozwala na jednoczesne przetwarzanie wielu połączeń. Przy każdym połączeniu następuje wywołanie zwrotne, a w przypadku jeśli nie będzie żadnej pracy do wykonania, Node.js przejdzie w tryb uśpienia.
\footcite{Node.js2017}

\par Środowisko Node.js oparte jest na implementacji silnika ``V8'' stworzonego przez Google. Zaimplementowany głownie jest w języku C i C++, koncentrując się na wydajności i niskim zużyciu pamięci. Różnica polega na tym, że silnik ``V8'' obsługuje głównie JavaScript w przeglądarkach internetowych, a Node.js został stworzony z myślą o obsłudze długotrwałych procesów serwerowych.
\par W celu obsługi jednoczesnego wykonywania logiki biznesowej, Node.js opiera się na asynchronicznym modelu zdarzeń wejścia i wyjścia, w przeciwieństwie do większości innych współczesnych środowisk, które oparte są na wielowątkowości. Model zdarzeń jest obsługiwany na poziomie języka, a jest to możliwe ponieważ JavaScript obsługuje wywołania zwrotne zdarzeń oraz funkcjonalny charakter JavaScript sprawia, że niezwykle łatwo jest tworzyć anonimowe obiekty funkcji, które można zarejestrować jako programy obsługi zdarzeń. \footcite{Tilkov2010}

\subsection{.NET Framework}
\par Jest to platforma programistyczna, która pozwala na tworzenie i uruchamianie aplikacji i usług dla systemu operacyjnego Windows. Zawiera wiele narzędzi pozwalających na tworzenie aplikacji w różnych językach programowania, które mogą być następnie uruchamiane na dostarczonej maszynie wirtualnej. \footcite{dotNetFr}

\par Architektura środowiska .NET składa się z takich komponentów jak:
\begin{itemize}
  \item CTS (Common Type System) - opisuje wszystkie wspierane przez platformę typy. Definiuje zasady korzystania z danych typów, dostarcza zorientowany obiektowo model dla różnych języków implementowanych w .NET oraz zapewnia bibliotekę zawierającą prymitywne typy danych (takich jak \texttt{Boolean}, \texttt{char} itp.).
  \item CLS (Common Language Specyfication) - definiuje w jaki sposób mają być definiowane obiekty i funkcje, w języku przeznaczonym na platformę .NET. CLS jest podzbiorem CTS, co oznacza, że wszystkie opisane zasady CTS dotyczą również CLS.\footcite{dotNetCLS}
  \item FCL (Framework Class Library) - jest to standardowa biblioteka zawierająca podstawę implementacji klas, interfejsów, typów wartości czy usług, które wykorzystywane są do tworzenia aplikacji.\footcite{dotNetFCL}
  \item CLR (Common Language Runtime) - jest to środowisko uruchomieniowe, które uruchamia kod i zapewnia usługi ułatwiające proces programowania. Środowisko wykonawcze automatycznie obsługuje układ obiektów i zarządza referencjami no nich, zwalniając je w przypadku kiedy już nie są używane.\footcite{dotNetCLR}
\end{itemize}

\subsection{IL Assembler}
\par Każdy z kompilatorów przeznaczonych na platformę .NET, bez względu na wybrany język, kompiluje kod do postaci pośredniej, jakim jest kod IL.
\par Wykonywalny kod IL jest w formacie binarnym i nie jest czytelny dla człowieka. Oczywiście jak inne wykonywalne kody binarne, mogą zostać przedstawione w postaci assemblera, tak i kod IL może zostać zaprezentowany w postaci IL Assemblera. Zestaw instrukcji jest taki jak w przypadku tradycyjnego assemblera. Przykładowo, aby dodać dwie liczby należy użyć instrukcji \texttt{add}, a w przypadku odejmowania, należy użyć instrukcji \texttt{sub}.
\par Środowisko uruchomieniowe .NET nie potrafi jednak odczytywać bezpośrednio IL Assemblera. Aby kod napisany w IL Assemblerze można było uruchomić, trzeba skompilować go do postaci binarnej IL. \footcite{ILAsm1}

\section{Technologie pokrewne}
Aktualnie istnieje wiele rozwiązań przetwarzających język JavaScript jak i narzędzi, dzięki którymi można budować, jak i uruchamiać aplikacje dedykowane na platformę .NET oraz środowiska uruchomieniowe JavaScript. W tym rozdziale zostaną opisane niektóre z tych technologii.

\subsection{JScript}
\par Jest to język programowania opracowany przez firmę \textbf{Microsoft} w oparciu o wczesne standardy języka JavaScript. Podstawową i największą różnicą, która wyróżnia język JScript jest to, że nie jest to prosty język skryptowy. Pozwala on na tworzenie w pełni funkcjonalnych aplikacji jako pliki wykonywalne, które można bezpośrednio uruchamiać na komputerach klienckich w środowisku uruchomieniowym .NET. \footcite{jscript}

\subsection{TypeScript}
\par Jest to język programowania o otwartym kodzie, który opiera się na JavaScript. Wzbogaca składnię JavaScript o statyczne definicje typów. Definiowane typy umożliwiają opisanie kształtu obiektów, tworzenie lepszej dokumentacji kodu oraz sprawdzanie poprawności kodu jeszcze przed jego uruchomieniem. 
\par Kod TypeScript jest przekształcany na kod JavaScript przy pomocy kompilatora \textit{TypeScript} lub \textit{Babel}. Przekształcony kod może być uruchamiany jako zwykły JavaScript na przeglądarkach czy też na maszynie wirtualnej Node.js. \footcite{typescript}

\subsection{Babel}
\par Jest to transkompilator języka JavaScript pozwalający na transpilacje nowych funkcjonalności do starego standardu. Dzięki temu narzędziu można między innymi na uruchomienie kodu zarówno w nowych, jak i starych przeglądarkach. Zawiera wiele opcji konfiguracji, które poza zmianą samych funkcjonalności \textit{JavaScript}, pozwala również zamienić kod \textit{TypeScript} czy kod \textit{React} do kodu \textit{JavaScript}. \footcite{babel}

\subsection{Deno}
\par Jest to środowisko uruchomieniowe pozwalające uruchamiać kod dla JavaScript oraz TypeScript. Stworzone zostało w oparciu o silnik \textit{JavaScript V8} oraz języku programowania \textit{Rust}. Deno jest oprogramowaniem o otwartym kodzie, które ma być wydajnym i bezpiecznym środowiskiem skryptowym. Jest to alternatywa dla środowiska \textit{Node.js}. \footcite{deno}

\subsection{Emscripten}
\par Jest to kompilator o otwartych źródłach, pozwalający kompilować kod w języku \textit{C\\C++} do \textit{WebAssembly} oraz uruchamiać go w przeglądarkach internetowych, \textit{Node.js} lub innych środowiskach uruchomieniowych \textit{WebAssembly}. 
\par Praktycznie każda przenośna baza kodu \textit{C} lub \textit{C++} może zostać skompilowana do WebAssembly przy użyciu \textit{Emscripten}, począwszy od gier o wysokiej wydajności, które muszą renderować grafikę, odtwarzać dźwięki oraz ładować i przetwarzać pliki, aż po frameworki aplikacji, takie jak \textit{Qt}. \footcite{emscripten}

\subsection{Mono}
\par Jest to platforma programistyczna o otwartych kodach źródłowych, która wzorowana jest na platformie \textit{.NET Framework}. Umożliwia tworzenie aplikacji międzyplatformowych. Implementacja platform \textit{.NET} w \textit{Mono} opiera się na standardach \textit{ECMA} dla \textit{C\#} i \textit{Common Language Infrastructure}. Tak jak platforma \textit{.NET Framework} dostarcza kompilator, gotowe biblioteki oraz środowisko uruchomieniowe. \footcite{mono}

\subsection{DotGNU}
\par Projekt \textit{GNU} mający na celu stworzenie i rozwijanie implementacji platformy \textit{.NET}, która będzie stanowić wolne oprogramowanie. Narzędzia \textit{DotGNU} są tworzone zgodnie ze standardami \textit{ECMA}, dając możliwość, swobodnego tworzenia i uruchamiania aplikacji w technologii \textit{.NET}. Platforma \textit{DotGNU} jest bardzo podobna do platformy \textit{Mono}. \footcite{DotGNU}

 % chapter1.tex zawiera treść rozdziału 1

%----------------------------------------------------------------------------------------
%	ROZDZIAŁ 2
%----------------------------------------------------------------------------------------
\chapter{Projekt kompilatora}
\label{rozdzial2}
Opis 
\section{Środowisko i narzędzia}
Opis sprzętu na którym będzie wszystko uruchomiane.
Do implementacji będą wykorzystane:

\begin{itemize}
  \item C\#
  \item Visual Studio Code
  \item WSL (Ubuntu 20.04 LTS)
  \item JavaScript
  \item Node.js
  \item mono-devel (ilasm, ildasm)
\end{itemize}

\section{Analiza języka JavaScript i określenie zakresu implementacji}
% Opis składni JavaScript.
% Implementacja JavaScript w standardzie ES5.
% Musi być Turing-complete. Dodatkowo implementacja funkcji.

\par W tym rozdziale zawarty zostanie zakres implementacji oraz opis poszczególnych elementów języka JavaScript. W projekcie zostanie zaimplementowana jedynie część standardu ECMAScript, a niektóre mechanizmy zostaną uproszczone.

% #PYTANIE Czy warto opisać rzeczy które nie będą implementowane?

\subsection{Wyrażenia}

\par Składnia języka JavaScript zapożycza wiele rozwiązań użytych w Javie, jednak na konstrukcję miły też wpływ takie języki jak: Awk, Perl i Python. W języku JavaScript instrukcję nazywane są wyrażeniami, które rozdzielane są znakiem średniaka. Znaki białe takie jak spacja, tabulator czy znak końca linii nie mają wpływu na sposób działania kolejnych elementów wyrażeń, stanowią jedynie sposób ich oddzielenia. W kodzie źródłowym JavaScript rozróżnialna jest wielkość liter oraz wspierany jest standard znaków Unicode. ECMAScript definiuje również zestaw słów kluczowych i literałów oraz zasady automatycznego umieszczania średników ASI (Automatic semicolon insertion).

\subsection{Komentarze}

\par Rozróżniane są dwa typy komentarzy:
\begin{enumerate}
  \item Jednoliniowy - definiowany jest przy pomocy znaku ``$//$'' oraz umieszczany jest na końcu linii.
  \begin{lstlisting}[language=JavaScript, caption=Przykład komentarza jednoliniowego, label=alg:kod1]
    console.log(); // komentarz
  \end{lstlisting}
  \item Wieloliniowy - zawarty jest pomiędzy dwoma elementami ``/*'' oraz ``*/''
  \begin{lstlisting}[language=JavaScript, caption=Przykład komentarza wieloliniowego, label=alg:kod2]
    console.log();
    /*
      komentarz na
      wiele linii
    */
  \end{lstlisting}
\end{enumerate}

\subsection{Deklaracje zmiennych i stałych}
\par Zmienne deklaruje się przy pomocy słów kluczowych \texttt{var}, \texttt{let} oraz \texttt{const}. Deklaracja przy pomocy \texttt{var} jest podstawowym sposobem tworzenia zmiennych w JavaScript. Zasięg takiej zmiennej nie może być ograniczony przez blok w którym jest zawarta, przez co może powodować błędy przy pisaniu kodu. W celu uściślenia zasięgu i przeznaczenia zmiennych powstały dwa inne sposoby deklaracji \texttt{let} oraz \texttt{const}. Oba te rodzaje deklaracji powodują, że zakres dostępności zmiennej jest ograniczony do bloku w którym została zadeklarowana. Różnicą między tymi dwoma deklaracjami jest taka, że przy pomocy \texttt{const} definiujemy stałą która musi być od razu zadeklarowana, a \texttt{let} działa podobnie jak \texttt{var}.
\begin{lstlisting}
  var zmienna1;
  let zmienna2;
  const stala = true;
\end{lstlisting}
\par Przy deklaracji zmiennych przy użyciu \texttt{var} lub \texttt{let}, których nie przypiszemy żadnej wartości to przyjmują one wartość \texttt{undefined}

\subsection{Typy danych}
\par W najnowszym standardzie ECMAScript zdefiniowanych jest siedem typów danych:
\begin{enumerate}
  \item \texttt{Boolean} - może przybierać dwie wartości \texttt{true} lub \texttt{false}.
  \item \texttt{null} - słowo kluczowe oznaczające wartość zerową. 
  \item \texttt{undefined} - wartość nieokreślona.
  \item \texttt{Number} - tym przeznaczony dla literałów całkowitych jak i zmiennoprzecinkowych.
  \item \texttt{String} - typ przeznaczony dla literałów łańcuchowych reprezentujących zero lub więcej pojedynczych znaków ujętych w podwójny lub pojedynczy cudzysłów.
  \item \texttt{Symbol} - wprowadzony w ECMAScript 6 typ danych, który pozwala na tworzenie unikalnych i nie zmiennych wartości.
  \item \texttt{Object} - typ złożony do którego zaliczają się funkcje, tablice, słowniki oraz instancje klas.
\end{enumerate}

\subsection{Operacje arytmetyczne}
\subsection{Operacje warunkowe}
\subsection{Zakres implementacji projektu}
W niniejszym projekcie zostaną zaimplementowane następujące elementy: 
\begin{itemize}
  \item Komentarze jednoliniowe oraz wieloliniowe.
  \item Proste typy danych: \texttt{Boolean}, \texttt{Number}, \texttt{String}.
  \item Tworzenie zmiennych typu \texttt{var}.
  \item Uproszczona implementacja funkcji \texttt{console.log()}.
  \item Konwersja typów danych.
  \item Operacje matematyczne takie jak dodawanie, odejmowanie, mnożenie oraz dzielenie.
  \item Wyrażenia warunkowe takie jak sprawdzenie: równości, nierówności, większości lub mniejszości.
  \item Instrukcja \texttt{if} oraz pętle \texttt{while} oraz \texttt{for}
  \item Typ \texttt{Object} pod postacią tablicy elementów oraz słownika danych.
  \item Deklaracja oraz wywoływanie funkcji: bez parametrów oraz zwracanej wartości, bez parametrów oraz z zwracaną wartością, z parametrami bez zwracanej wartości oraz z parametrami z zwracaną wartością.
\end{itemize}

% https://developer.mozilla.org/pl/docs/Web/JavaScript/Guide/Sk%C5%82adnia_i_typy

\section{Parser}
Używamy ANTLR z własną gramatyką ale posiłkując się gotowcem. Wykorzystanie gotowej gramatyki powoduje wygenerowanie tak dużych plików, że próba zrozumienia co jest do czego wymaga poświęcenia dużego wysiłku. Większość tych rzeczy i tak by nie została wykorzystana.

Rozważane możliwości i wykonano przegląd narzędzi:

po 2 zdania: \\
Gotowe narzędzia:
\begin{itemize}
  \item LEX \& YYAC
  \item ANTLR
  \item Coco/R
  \item gppg \& gplex
  \item Owl (https://github.com/ianh/owl)
\end{itemize}
i więcej... https://en.wikipedia.org/wiki/Comparison\_of\_parser\_generators


\section{Struktura projektu}
Diagramy i opisy.
Jak będzie wyglądał ten rozdział zależy jak wyjdzie implementacja.
% Tak, tak, wiem, najpierw implementacja a później dokumentacja...
 % chapter2.tex zawiera treść rozdziału 2

\chapter{Implementacja aplikacji kompilatora}
\label{rozdzial3}

\section{Parser}
Sposób implementacji lub przygotowania i użycia gotowych narzędzi.

\subsection{Analiza leksykalna}
Tekst
\subsection{Gramatyka}
Tekst
\section{Funkcjonalności}
Opis sposobu przetwarzania instrukcji
\section{Generowanie assemblera}
Opisać jak wygenerować kod .NET


\chapter{Testy}
\label{rozdzial4}

\par Podczas procesu tworzenia kompilatora, były również tworzone programy testujące dla poszczególnych funkcjonalności, w celu zweryfikowania prawidłowego działania programu. Dla każdego z modułów został utworzony program testujący w języku JavaScript, który obejmuje zakres funkcjonalności modułu. Zostały również utworzone tożsame programy w języku C\# w celu porównania do kodu języka JavaScript.
\par Zostały również wykorzystane dwa gotowe programy napisane w JavaScript odnalezione w Internecie. Również dla tych programów zostały utworzone odpowiedniki w języku C\# oraz zostały stworzone programy w języku JScript. Dla programów zostały przeprowadzone dodatkowe testy: został zmierzony czas wykonywania programu, zużycie pamięci oraz wielkości pliku wynikowego.

\section{Testy modułów}

\par A

\subsection{Porównanie wyniku wykonania programów}

\par Pierwszym z testów został wykonany program testujący wyświetlanie elementów na ekranie. Program wyświetla różne wartości różnych typów, takie jak wartości tekstowe w cudzysłowu podwójnym jak i pojedynczym, wartości liczbowe całkowite i rzeczywiste, wartości logiczne oraz tablice wartości. Na załączonym rysunku \ref{fig:result1} przedstawione są zrzuty ekranu wyniku działania programów w różnych wariantach.

\begin{figure}[!h]
	\centering
  \includegraphics[width=0.7\linewidth]{t1.png}
	\caption{Wyniki wykonania programu testowego dla modułu obsługi standardowego wyjścia. Źródło: własne}
	\label{fig:result1}
\end{figure}

\par Jak można zauważyć, wynik programu w języku C\# oraz JavaScript kompilowanych na wspólną platformę .NET są identyczne. Tutaj należy również wspomnieć, że standardowa biblioteka C\# nie posiada, żadnej z funkcji pozwalającej na wyświetlenie listy elementów tablicy przy pomocy jednej instrukcji. Tak jak podczas implementacji kompilatora JavaScript została zastosowana tutaj funkcja łącząca elementy tablicy w jeden ciąg znaków oraz zostały doklejone nawiasy otwierające oraz zamykające. 
\par Porównując wynik programu uruchomionego na platformie .NET z programem uruchomionym na platformie Node.js, można zauważyć nie wielkie różnice w wyświetlanych wynikach. Pierwszą z nich rzucającą się w oczy, jest zmiana koloru wyświetlanego tekstu dla liczb oraz wartości liczbowych. Jednak ten efekt zależny jest od formatowania kolorów w danej konsoli. Konsola wykryła wywołanie platformy Node.js, dzięki czemu zmieniła kolorystykę.
\par Drugim z różniącym się elementem jest sposób wyświetlania liczb rzeczywistych. Różnią się tym, że dla platformy .NET wyświetlany jest przecinek, kiedy na platformie Node.js wyświetlana jest kropka. Ostatnią różnicą w prezentowanych wynikach jest wyświetlanie wartości logicznych. Dla platformy .NET wartości są wyświetlane z wielką literą, a w przypadku Node.js wartości wyświetlane są małymi literami.

\par Kolejnym z testów dla którego wynik wykonania programu był różny, został przeprowadzony dla modułu działań arytmetycznych. Zrzuty wyników zostały przedstawione na rysunku \ref{fig:result2}. Głównie różnice występują przy wyświetlaniu wyniku operacji dzielenia.

\begin{figure}[!h]
	\centering
  \includegraphics[width=0.7\linewidth]{t2.png}
	\caption{Wyniki wykonania programu testowego dla modułu obsługi działania arytmetyczne. Źródło: własne}
	\label{fig:result2}
\end{figure}

\par Pierwszą z różnic między programem napisanym w języku C\# a JavaScript na platformie .NET jest wynik dzielenia dwóch liczb całkowitych. Wykonywane działanie ma postać $y = 10 / 15$, więc wynikiem jest liczba rzeczywista. W programie w języku C\# jest językiem silnie typowanym, co oznacza, że jeżeli przynajmniej jedna z liczb nie zostanie przekonwertowana na typ liczb rzeczywistych, to wynik będzie typu liczby całkowitej. W tworzonym kompilatorze została zaimplementowana automatyczna konwersja typów, w momencie kiedy zostanie wykryta operacja dzielenia dwóch liczb.
\par Drugą z widocznych różnic jest precyzja wartości zmiennych. W tworzonym kompilatorze JavaScript został wykorzystany typ pojedynczej precyzji, a w przypadku kompilatora C\# została zastosowana typ podwójnej precyzji. Porównując wyniki platformy .NET do platformy Node.js można zauważyć, że wartości na platformie .NET są zaokrąglane.

\par Przy uruchomieniu programu JScript jednego z algorytmów testujących, ukazała się jeszcze jedna różnica w prezentowanych na konsoli wynikach. Różnicą jest sposób wyświetlania tablicy elementów. W tworzonym kompilatorze, jak i na platformie .NET, elementy są otoczone nawiasem kwadratowym, oraz oddzielone są poza przecinkiem dodatkową spacją. W przypadku wyniku programu JScript, prezentowana tablica nie posiada nawiasów kwadratowych oraz liczby są oddzielone przecinkami bez spacji. Wynik programów zaprezentowany jest na rysunki \ref{fig:result3}

\begin{figure}[!h]
	\centering
  \includegraphics[width=0.7\linewidth]{t3.png}
	\caption{Wyniki wykonania programu dla algorytmu testowego nr 1. Źródło: własne}
	\label{fig:result3}
\end{figure}

\par Skrypty testowe dla pozostałych modułów nie wykazują różnic w prezentowanych wynikach na konsoli.

\subsection{Porównanie generowanego kodu assemblera}

\par Generowany kod assemblera z języka JavaScript będzie porównany z kodem dezasemblowanym kodem programu napisanego w języku C\#. Dezasemblacja jest wykonywana przy pomocy programu \textit{ildasm.exe} znajdujący się w pakiecie \textbf{.NET Framework}.
\par Pierwszą z widocznych różnic jest ilość deklaracji metadanych w nagłówku pliku. Kolejną z różnic jest ilość modyfikatorów przy deklaracji klasy jak i metody \texttt{Main}. Dla deklaracji klasy programu C\# zostały użyte następujące słowa kluczowe:
\begin{itemize}
	\item \texttt{private}
	\item \texttt{auto}
	\item \texttt{ansi}
	\item \texttt{beforefieldinit}
\end{itemize}


% 1. Obsługa standardowego wyjścia
% 2. Obsługa zmiennych
% 3. Obsługa działań arytmetycznych
% 4. Obsługa wyrażeń warunkowych
% 5. Obsługa pętli
% 6. Obsługa tablic
% 7. Obsługa funkcji

% \section{Testy algorytmów}

% \section{Proste operacje matematyczne}
% Test dodawania, odejmowania, mnożenia, dzielenia, przypisywania.
% \section{Kolejność wykonywania działań}
% Test na bardziej złożonych wyrażeniach. Sprawdzenie poprawności działania nawiasów oraz kolejności wykonywania działań.
% \section{Wyrażenia warunkowe}
% Test wyrażeń warunkowych. Kolejność wykonywania operacji and i or.
% \section{Tablice}
% Test obsługi tablic jedno i wielowymiarowych.
% \section{Obiekty}
% Test obsługi obiektów.
% \section{Klasy}
% Jeśli będzie implementacja.
% Test działania obiektów klas.
% \section{Funkcje}
% Test działania funkcji.
% 1. Funkcja "void" bez parametrów.
% 2. Funkcja "void" z parametrami.
% 3. Funkcja zwracająca różne typy (proste, tablice, obiekty) bez parametrów.
% 4. Funkcje zwracająca różne typy z parametrami.
% 5. inne

% \section{Algorytm 1}
% \subsection{Opracowanie pseudokodu algorytmu 1}
% \subsection{Implementacja algorytmu 1}
% \subsection{Testy algorytmu 1}
% \section{Algorytm 2}
% \subsection{Opracowanie pseudokodu algorytmu 2}
% \subsection{Implementacja algorytmu 2}
% \subsection{Testy algorytmu 2}


%----------------------------------------------------------------------------------------
%	ROZDZIAŁy kolejne należy dodać analogicznie do 1 i 2 
% utworzyć pliki i je załączyć (include)
%----------------------------------------------------------------------------------------

%----------------------------------------------------------------------------------------
%	ZAKOŃCZENIE PRACY DYPLOMOWEJ (JK - design and implementation)
%----------------------------------------------------------------------------------------
\addcontentsline{toc}{chapter}{Podsumowanie}
%%%%%%%%%%%%%%%%%%%%%%%%%%%%%%%%%%%%%%%%%
% Wnioski do pracy dyplomowej
% Szablon pracy dyplomowej
% Wydział Informatyki 
% Zachodniopomorski Uniwersytet Technologiczny w Szczecinie
% autor Joanna Kołodziejczyk (jkolodziejczyk@zut.edu.pl)
% Bardzo wczesnym pierwowzorem szablonu był
% The Legrand Orange Book
% Version 2.1 (26/09/2018)
%
% Modifications to LOB assigned by %JK
%%%%%%%%%%%%%%%%%%%%%%%%%%%%%%%%%%%%%%%%%


\chapter*{Podsumowanie}

\par Celem niniejszej pracy było zaprojektowanie oraz implementacja kompilatora języka JavaScript na kod IL Assembler uruchamianego na platformie .NET. Praca przedstawia dostępne narzędzia, technologie oraz języki programowania wykorzystane w projekcie oraz technologie pokrewne które mogły by być wykorzystane.
\par W pracy została przeprowadzona częściowa analiza języka JavaScript na podstawie której został ustalony zakres implementacji kompilatora. Zostało również zaprojektowany oraz wyjaśniony sposób implementacji tworzonego kompilatora.
\par W celu zweryfikowania poprawności działania kompilatora, zostały stworzone skrypty testujące dla poszczególnych funkcjonalności oraz ich odpowiedniki w języku C\#. Sposób działania utworzonego pliku wykonywalnego przez kompilator został porównany do wywołania danego skryptu testowego na platformie Node.js oraz do wywołania odpowiednika kodu w C\# na platformie .NET. Przy analizie wyników poszczególnych testów funkcjonalności zostały wykazane niewielkie różnice przy wyświetlaniu elementów na konsoli. Zostało również przeprowadzone porównanie generowanego kodu assemblera, które wykazało słabą optymalizację stworzonego kompilatora.
\par Zostały również przeprowadzone testy na kodzie JavaScript odnalezionym w Internecie. Dla tych kodów zostały utworzone również odpowiedniki w języku C\# oraz JScript. Programy zostały przetestowane pod względem szybkości działania, zużycia pamięci oraz wykorzystania przestrzeni dyskowej. Testy wykazały, że utworzony kompilator osiąga porównywalne wyniki dla kodu kompilowanego z języka C\# oraz lepsze wyniki w porównaniu do kodu JScript.
\par Kolejnym krokiem rozwijającym stworzony kompilator jest zaimplementowanie pełnej funkcjonalności języka JavaScript oraz zastosowanie pełnej gamy instrukcji asemblerowych udostępnianych przez środowisko .NET. Dzięki temu można by było sprawdzić kompilator na bardziej złożonych programach JavaScript, czy też dostępnych zewnętrznych bibliotekach oraz frameworków dla tego języka.

% Podsumowanie pracy powinno na maksymalnie dwóch stronach przedstawić główne wyniki pracy dyplomowej. Struktura zakończenia to:
% \begin{enumerate}
% \item Przypomnienie celu i hipotez
% \item Co w pracy wykonano by cel osiągnąć (analiza, projekt, oprogramowanie, badania eksperymentalne)
% \item Omówienie głównych wyników pracy
% \item Jak wyniki wzbogacają dziedzinę
% \item Zamknięcie np. poprzez wskazanie dalszych kierunków badań.
% \end{enumerate}
 % conclusions.tex zawiera treść zakończenia/podsumowania pracy dyplomowej


%----------------------------------------------------------------------------------------
%	SPIS LITERATURY (JK - design and implementation)
%----------------------------------------------------------------------------------------
\pagestyle{empty} 
\chapter*{Spis literatury}
\addcontentsline{toc}{chapter}{\textcolor{blueZUT}{Spis literatury}}
% Poniżej zdefiniowane są filtry do dzielenia spisu literatury na kategorie
\defbibfilter{articles}{
  type=article or 
  type=inproceedings
}
\defbibfilter{other}{
  type=url or
  type=online  or
  type=manual or
  type=misc
}

% UWAGA! -aby zmienić zawartość spisu literatury należy wyedytować plik
% bibliography.bib

%------------------------------------------------
% Spis literartury podzielony jest na 3 kategorie
% 1
\section*{Książki}
\addcontentsline{toc}{section}{Książki}
\printbibliography[heading=bibempty,type=book]

%------------------------------------------------
% 2
\section*{Artykuły}
\addcontentsline{toc}{section}{Artykuły}
\printbibliography[heading=bibempty,filter=articles]
%------------------------------------------------
% 3
\section*{Źródła internetowe i inne}
\addcontentsline{toc}{section}{Źródła internetowe i inne}
\printbibliography[heading=bibempty,filter=other]


%----------------------------------------------------------------------------------------
%	APPENDIX (JK - design and implementation)
%----------------------------------------------------------------------------------------
\pagestyle{fancy} 
\begin{appendix}
\appendix
\include{appendixA} % conclusions.tex zawiera treść zakończenia/podsumowania pracy dyplomowej
\end{appendix}
%----------------------------------------------------------------------------------------
% Koniec dokumentu
\end{document}
