%%%%%%%%%%%%%%%%%%%%%%%%%%%%%%%%%%%%%%%%%
% Plik konfigurujący
% Szablon pracy dyplomowej
% Wydział Informatyki 
% Zachodniopomorski Uniwersytet Technologiczny w Szczecinie
% autor Joanna Kołodziejczyk (jkolodziejczyk@zut.edu.pl)
% Bardzo wczesnym pierwowzorem szablonu był
% The Legrand Orange Book
% Version 2.1 (26/09/2018)
%
% Modifications to LOB assigned by %JK
%%%%%%%%%%%%%%%%%%%%%%%%%%%%%%%%%%%%%%%%%


%----------------------------------------------------------------------------------------
%	VARIOUS REQUIRED PACKAGES AND CONFIGURATIONS
%----------------------------------------------------------------------------------------

%GEOMETRY
\usepackage[top=3.2cm,bottom=3.5cm,left=2.5cm,right=2.5cm,headsep=1.5ex,bindingoffset=1cm,a4paper]{geometry} % Page margins

% GRAPHICS
\usepackage{graphicx} % Required for including pictures
\graphicspath{{Pictures/}} % Specifies the directory where pictures are stored

% For math equations, theorems, symbols, etc
\usepackage{amsmath,amsfonts,amssymb,amsthm} 

% Customize lists
\usepackage{enumitem} 
\setlist{nolistsep} % Reduce spacing between bullet points and numbered lists
\setlist[itemize]{label=--}

% Required for nicer horizontal rules in tables
\usepackage{booktabs} 

\usepackage{xcolor} % Required for specifying colors by name
% Define blue colors used for highlighting throughout the book- based on the WI ZUT colors
\definecolor{blueWI}{cmyk}{.6,.2,0,.0} % JK - Define the blue colour used for highlighting throughout the book
\definecolor{blueZUT}{cmyk}{1,.75,0,.2} % JK - Define the blue colour used for highlighting throughout the book
\definecolor{grayZUT}{cmyk}{0,0,0,0.4} % JK - Define the blue colour used for highlighting throughout the book

%other definision
\newcommand{\rulecolor}[1]{\color{#1}\rule}

%----------------------------------------------------------------------------------------
%	FONTS AND LANGUAGE (JK - configuring and styling)
%----------------------------------------------------------------------------------------
%
\usepackage{newtxmath,newtxtext}
\usepackage{t1enc}
\usepackage[polish]{babel}% moved after to avoid conflict between polish babel and amsmath
\usepackage[utf8]{inputenc} 
%\usepackage{avant} % Use the Avantgarde font for headings

%----------------------------------------------------------------------------------------
%	PAGE HEADERS AND FOOTERS (JK - Styling for the current chapter in the header)
%----------------------------------------------------------------------------------------

\usepackage{fancyhdr} % Required for header and footer configuration
\setlength{\headheight}{2.5ex}
\pagestyle{fancy} % Enable the custom headers and footers

\renewcommand{\chaptermark}[1]{\markboth{\sffamily\normalsize\thechapter.\hspace{5pt} #1}{}} % JK - Styling for the current chapter in the header
\renewcommand{\sectionmark}[1]{\markright{\sffamily\normalsize\thesection\hspace{5pt} #1}{}} % Styling for the current section in the header

\fancyhf{} % Clear default headers and footers

% JK - header with page hanging  and chapter title in the box
\fancyhead[EL]{%
  \textcolor{white}{%
    \llap{%
      \colorbox{blueZUT}{%
        \makebox[7ex][r]{\sffamily\thepage}%
      }%
      \hspace{1.25\marginparsep}%
      \hspace{-\fboxsep}%
    }%
    \textcolor{black}{%
      \colorbox{blueWI!20}{%
        \makebox[\textwidth][l]{\sffamily\shorttitle}}}
     \addtolength{\headheight}{5ex} % Increase the spacing around the header slightly
  }%
}
\fancyhead[OR]{%
  \textcolor{white}{%
    \llap{%
      \colorbox{blueZUT}{%
        \makebox[7ex][r]{\sffamily\thepage}%
      }%
      \hspace{1.25\marginparsep}%
      \hspace{-\fboxsep}%
    }%
    \textcolor{black}{%
      \colorbox{blueWI!20}{%
        \makebox[\textwidth][l]{\sffamily\leftmark}}}%{\rightmark}}}
    \addtolength{\headheight}{5ex} % Increase the spacing around the header slightly    
  }%
}

\fancypagestyle{plain}{%
   \fancyhead{} % get rid of headers
   \fancyfoot[RE,RO]{
  	\textcolor{white}{%
    	\llap{%
      	\colorbox{blueZUT}{%
        	\makebox[7ex][l]{\sffamily\thepage}%
      }%
      \hspace{-6\marginparsep}%separation from the margin
      \hspace{-5\fboxsep}%
     }%
     }%
    } 
   \addtolength{\headheight}{18pt} % Increase the spacing around the header slightly
}


\renewcommand*{\headrulewidth}{0pt}
\renewcommand*{\footrulewidth}{0pt}

% Removes the header from odd empty pages at the end of chapters
\makeatletter
\renewcommand{\cleardoublepage}{
\clearpage\ifodd\c@page\else
\hbox{}
\vspace*{\fill}
\thispagestyle{empty}
\newpage
\fi}

%----------------------------------------------------------------------------------------
%	BIBLIOGRAPHY (JK - configuring and styling)
%----------------------------------------------------------------------------------------
% 
\usepackage[
style=numeric,% style alphabetic or numeric
citestyle=verbose,
sorting=nyt,%name -year -title
sortcites=true,
autopunct=true,
autolang=hyphen,
hyperref=true, % if the citation is the link to bibliography
backend=biber,
defernumbers=true]{biblatex}
\addbibresource{bibliography.bib} % BibTeX bibliography file
\defbibheading{bibempty}{}
\nocite{*}

%\usepackage{calc} % For simpler calculation - used for spacing the index letter headings correctly

%----------------------------------------------------------------------------------------
%	CHAPTER & SECTION HEADINGS
%----------------------------------------------------------------------------------------

\usepackage[explicit]{titlesec}

% \titleformat{<command>}[<shape>]{<format>}{<label>}{<sep>}{<before-code>}[<after-code>]

\titleformat{\chapter}[block]
  {\huge\sffamily\color{blueZUT}}
  {\hspace{- 3ex}{\thechapter.} \hspace{0.5em}{#1\strut}}
  {0pt}
  {}

\titleformat{name = \chapter, numberless}[block]
  {\huge\sffamily\color{blueZUT}}
  {{#1\strut}}
  {0pt}
  {}

%----------------------------------------------------------------------------------------
%	Hanging SECTION NUMBERING (MARGIN)
%----------------------------------------------------------------------------------------

\makeatletter
\renewcommand{\@seccntformat}[1]{\llap{\textcolor{blueZUT}{\csname the#1\endcsname}\hspace{1em}}}                    

\renewcommand{\section}
{\@startsection{section}{1}{\z@}
{-4ex \@plus -1ex \@minus -.4ex}
{1ex \@plus.2ex }
{\normalfont\large\sffamily\bfseries}}

\renewcommand{\subsection}{\@startsection {subsection}{2}{\z@}
{-3ex \@plus -0.1ex \@minus -.4ex}
{0.5ex \@plus.2ex }
{\normalfont\sffamily\bfseries}}

\renewcommand{\subsubsection}{\@startsection {subsubsection}{3}{\z@}
{-2ex \@plus -0.1ex \@minus -.2ex}
{.2ex \@plus.2ex }
{\normalfont\small\sffamily\bfseries}}                        

\renewcommand\paragraph{\@startsection{paragraph}{4}{\z@}
{-2ex \@plus-.2ex \@minus .2ex}
{.1ex}
{\normalfont\small\sffamily\bfseries}}



%----------------------------------------------------------------------------------------
%	MAIN TABLE OF CONTENTS (JK modification: style, indentation, colors,)
%----------------------------------------------------------------------------------------

\usepackage{titletoc} % Required for manipulating the table of contents
\contentsmargin{0cm} % Removes the default margin

% Chapter text styling
\titlecontents{chapter}[1.25cm] % Indentation
{\addvspace{12pt}\large\sffamily} % Spacing and font options for chapters
{\color{blueZUT!60}\contentslabel[\Large\thecontentslabel]{1.25cm}\color{blueZUT}} % JK Chapter number
{\color{blueZUT}}  
{\color{blueZUT!60}\normalsize\;\titlerule*[.5pc]{.}\;\thecontentspage} % Page number

% Section text styling
\titlecontents{section}
	[1.25cm] % Left indentation
	{\addvspace{3pt}\sffamily} % Spacing and font options for sections
	{\contentslabel[\thecontentslabel]{1.25cm}} % Formatting of numbered sections of this type
	{} % Formatting of numberless sections of this type
	{\;\titlerule*[.5pc]{.}\;\color{black}\thecontentspage} % Formatting of the filler to the right of the heading and the page number

% Subsection text styling
\titlecontents{subsection}
	[2.5cm] % Left indentation
	{\addvspace{1pt}\sffamily\small} % Spacing and font options for subsections
	{\contentslabel[\thecontentslabel]{1.25cm}} % Formatting of numbered sections of this type
	{} % Formatting of numberless sections of this type
	{\ \titlerule*[.5pc]{.}\;\thecontentspage} % Formatting of the filler to the right of the heading and the page number

%%%%% JK Add figures and tables
% Figure text styling
\titlecontents{figure}
	[0em] % Left indentation
	{\addvspace{3pt}\sffamily} % Spacing and font options for figures
	{\contentslabel[\thecontentslabel]{1.25cm}} % Formatting of numbered sections of this type
	{} % Formatting of numberless sections of this type
	{\ \titlerule*[.5pc]{.}\;\thecontentspage} % Formatting of the filler to the right of the heading and the page number

% Table text styling
\titlecontents{table}
	[0em] % Left indentation
	{\addvspace{3pt}\sffamily} % Spacing and font options for tables
	{\contentslabel[\thecontentslabel]{1.25cm}} % Formatting of numbered sections of this type
	{} % Formatting of numberless sections of this type
	{\ \titlerule*[.5pc]{.}\;\thecontentspage} % Formatting of the filler to the right of the heading and the page number


%----------------------------------------------------------------------------------------
%	THEOREM STYLES
%----------------------------------------------------------------------------------------

\newcommand{\intoo}[2]{\mathopen{]}#1\,;#2\mathclose{[}}
\newcommand{\ud}{\mathop{\mathrm{{}d}}\mathopen{}}
\newcommand{\intff}[2]{\mathopen{[}#1\,;#2\mathclose{]}}
\renewcommand{\qedsymbol}{$\blacksquare$}
\renewcommand{\thmname}{Twierdzenie}

% Boxed/framed environments
\newtheoremstyle{blueZUTnumbox}% Theorem style name
{0pt}% Space above
{0pt}% Space below
{\normalfont}% Body font
{}% Indent amount
{\small\bf\sffamily\color{blueZUT}}% Theorem head font
{\;}% Punctuation after theorem head
{0.25em}% Space after theorem head
{\small\sffamily\color{blueZUT}\thmname{#1}\nobreakspace\thmnumber{\@ifnotempty{#1}{}\@upn{#2}}% Theorem text (e.g. Theorem 2.1)
\thmnote{\nobreakspace\the\thm@notefont\sffamily\bfseries\color{black}---\nobreakspace#3.}} % Optional theorem note

% \newtheoremstyle{blacknumex}% Theorem style name
% {5pt}% Space above
% {5pt}% Space below
% {\normalfont}% Body font
% {} % Indent amount
% {\small\bf\sffamily}% Theorem head font
% {\;}% Punctuation after theorem head
% {0.25em}% Space after theorem head
% {\small\sffamily{\tiny\ensuremath{\blacksquare}}\nobreakspace\thmname{#1}\nobreakspace\thmnumber{\@ifnotempty{#1}{}\@upn{#2}}% Theorem text (e.g. Theorem 2.1)
% \thmnote{\nobreakspace\the\thm@notefont\sffamily\bfseries---\nobreakspace#3.}}% Optional theorem note

\newtheoremstyle{blacknumbox} % Theorem style name
{5pt}% Space above
{5pt}% Space below
{\normalfont}% Body font
{}% Indent amount
{\small\bf\sffamily}% Theorem head font
{\;}% Punctuation after theorem head
{0.25em}% Space after theorem head
{\small\sffamily\thmname{#1}\nobreakspace\thmnumber{\@ifnotempty{#1}{}\@upn{#2}}% Theorem text (e.g. Theorem 2.1)
\thmnote{\nobreakspace\the\thm@notefont\sffamily\bfseries---\nobreakspace#3.}}% Optional theorem note

% Non-boxed/non-framed environments
\newtheoremstyle{blueZUTnum}% Theorem style name
{5pt}% Space above
{5pt}% Space below
{\normalfont}% Body font
{}% Indent amount
{\small\bf\sffamily\color{blueZUT}}% Theorem head font
{\;}% Punctuation after theorem head
{0.25em}% Space after theorem head
{\small\sffamily\color{blueZUT}\thmname{#1}\nobreakspace\thmnumber{\@ifnotempty{#1}{}\@upn{#2}}% Theorem text (e.g. Theorem 2.1)
\thmnote{\nobreakspace\the\thm@notefont\sffamily\bfseries\color{black}---\nobreakspace#3.}} % Optional theorem note
\makeatother

% Defines the theorem text style for each type of theorem to one of the three styles above
\newcounter{dummy} 
\numberwithin{dummy}{section}
\theoremstyle{blueZUTnumbox}
\newtheorem{theoremeT}[dummy]{Twierdzenie}
\theoremstyle{blueZUTnum}
\newtheorem{exampleT}{Przykład}[chapter]
\theoremstyle{blacknumbox}
\newtheorem{definitionT}{Definicja}[section]


%----------------------------------------------------------------------------------------
%	DEFINITION OF COLORED BOXES
%----------------------------------------------------------------------------------------

\RequirePackage[framemethod=default]{mdframed} % Required for creating the theorem, definition, exercise and corollary boxes

% Theorem box
\newmdenv[skipabove=7pt,
skipbelow=7pt,
backgroundcolor=black!3,
linecolor=blueZUT,
innerleftmargin=5pt,
innerrightmargin=5pt,
innertopmargin=5pt,
leftmargin=0cm,
rightmargin=0cm,
innerbottommargin=5pt]{tBox}


% Definition box
\newmdenv[skipabove=7pt,
skipbelow=7pt,
rightline=false,
leftline=true,
topline=false,
bottomline=false,
linecolor=blueZUT,
innerleftmargin=5pt,
innerrightmargin=5pt,
innertopmargin=0pt,
leftmargin=0cm,
rightmargin=0cm,
linewidth=2pt,
innerbottommargin=0pt]{dBox}	

% Creates an environment for each type of theorem and assigns it a theorem text style from the "Theorem Styles" section above and a colored box from above
\newenvironment{theorem}{\begin{tBox}\begin{theoremeT}}{\end{theoremeT}\end{tBox}}				  
\newenvironment{definition}{\begin{dBox}\begin{definitionT}}{\end{definitionT}\end{dBox}}	
\newenvironment{example}{\begin{exampleT}}{\hfill{\tiny\ensuremath{\blacksquare}}\end{exampleT}}		



%----------------------------------------------------------------------------------------
%	LISTING ENVIRONMENT
%----------------------------------------------------------------------------------------

\usepackage{listings}

%Polish set of letteres accepted in the listings
\lstset{
literate=%
{ą}{{\k{a}}}1
{Ą}{{\k{A}}}1
{ć}{{\'c}}1
{Ć}{{\'{C}}}1
{ę}{{\k{e}}}1
{Ę}{{\k{E}}}1
{ł}{{\l{}}}1
{Ł}{{\L{}}}1
{ń}{{\'n}}1
{Ń}{{\'N}}1
{ó}{{\'o}}1
{Ó}{{\'O}}1
{ś}{{\'s}}1
{Ś}{{\'S}}1
{ż}{{\.z}}1
{Ż}{{\.Z}}1
{ź}{{\'z}}1
{Ź}{{\'Z}}1
}

\renewcommand{\lstlistingname}{\small\sffamily\bfseries\color{blueZUT} Algorytm} % Change default listing caption to Algorthm
\renewcommand{\lstlistlistingname}{Lista \lstlistingname ów}

\definecolor{codegreen}{rgb}{0,0.6,0}
\definecolor{codegray}{rgb}{0.5,0.5,0.5}

\lstdefinestyle{mystyle}{
 %   backgroundcolor=\color{grayZUT!10},   
    basicstyle= \scriptsize\fontfamily{lmss}\selectfont,%\footnotesize\fontfamily{cmss}\selectfont,
    commentstyle=\color{codegray},
    keywordstyle=\color{violet},
    numberstyle=\tiny\color{codegray},%numeracja linijek
    identifierstyle={\color{black}},
    numbers=left,%numeracja linijek
    numbersep=10pt,%numeracja linijek
    stringstyle=\color{codegreen},
    breakatwhitespace=true,         
    breaklines=true,                 
    captionpos=b,                    
    %keepspaces=false,                
    %showspaces=false,                
    showstringspaces=false,
    showtabs=true,                  
    tabsize=2,
    frame=leftline,
    rulecolor = \color{blueWI},
    xleftmargin=5ex,
    xrightmargin=5ex
    }
 
\lstset{style=mystyle}

%----------------------------------------------------------------------------------------
% CAPTIONS ( JK - design and implementation)
%----------------------------------------------------------------------------------------

\usepackage{caption}
\captionsetup[figure]{name={\small\sffamily\color{blueZUT} Rysunek}}
\captionsetup[table]{name={\small\sffamily\color{blueZUT} Tabela}}
\captionsetup{font={small,sf,singlespacing}}


%----------------------------------------------------------------------------------------
%	HYPERLINKS IN THE DOCUMENTS
%----------------------------------------------------------------------------------------

\usepackage{hyperref}
%\hypersetup{hidelinks,backref=true,pagebackref=true,hyperindex=true,colorlinks=false,breaklinks=true,urlcolor=blueZUT,bookmarks=true,bookmarksopen=false}
\hypersetup{hidelinks,breaklinks=true,urlcolor=blueZUT,bookmarksopen=false,pdftitle={Title},pdfauthor={Author}}

%----------------------------------------------------------------------------------------
%	Listings languages
%----------------------------------------------------------------------------------------


\lstdefinelanguage{JavaScript}{
  keywords={typeof, new, true, false, catch, function, return, null, catch, switch, var, if, in, while, do, else, case, break},
  keywordstyle=\color{blue}\bfseries,
  ndkeywords={class, export, boolean, throw, implements, import, this},
  ndkeywordstyle=\color{darkgray}\bfseries,
  identifierstyle=\color{black},
  sensitive=false,
  comment=[l]{//},
  morecomment=[s]{/*}{*/},
  commentstyle=\color{purple}\ttfamily,
  stringstyle=\color{red}\ttfamily,
  morestring=[b]',
  morestring=[b]"
}

\lstdefinelanguage{CSharp}
{
 morecomment = [l]{//}, 
 morecomment = [l]{///},
 morecomment = [s]{/*}{*/},
 morestring=[b]", 
 sensitive = true,
 morekeywords = {abstract,  event,  new,  struct,
   as,  explicit,  null,  switch,
   base,  extern,  object,  this,
   bool,  false,  operator,  throw,
   break,  finally,  out,  true,
   byte,  fixed,  override,  try,
   case,  float,  params,  typeof,
   catch,  for,  private,  uint,
   char,  foreach,  protected,  ulong,
   checked,  goto,  public,  unchecked,
   class,  if,  readonly,  unsafe,
   const,  implicit,  ref,  ushort,
   continue,  in,  return,  using,
   decimal,  int,  sbyte,  virtual,
   default,  interface,  sealed,  volatile,
   delegate,  internal,  short,  void,
   do,  is,  sizeof,  while,
   double,  lock,  stackalloc,   
   else,  long,  static,   
   enum,  namespace,  string}
}

\lstdefinelanguage{IL}{%
  % so listings can detect directives and register names
  alsoletter={.\$},
  % strings, characters, and comments
  morestring=[b]",
  morestring=[b]',
  morecomment = [l]{//}, 
  % instructions
  morekeywords={[1]ldstr, ldstr, call, callvirt, instance, ldloc, ldc.i4,
   stloc, conv.r4, ldc.r4, ldloca.s, mul, add, div, sub, ldc.i4.0, cgt, ceq,
   br, brfalse, newobj, .class, extends, .method, static, cil, managed, .entrypoint},
  % assembler directives
  morekeywords={[2]string, void, int32, float32, bool},
}[strings,comments,keywords]

\usepackage{xurl}

\usepackage{pgfplots}

\pgfplotsset{width=\linewidth,compat=1.9}

\usepackage{pdfpages}
