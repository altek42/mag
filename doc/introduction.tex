%%%%%%%%%%%%%%%%%%%%%%%%%%%%%%%%%%%%%%%%%
% Plik z wstępem do pracy
% Szablon pracy dyplomowej
% Wydział Informatyki 
% Zachodniopomorski Uniwersytet Technologiczny w Szczecinie
% autor Joanna Kołodziejczyk (jkolodziejczyk@zut.edu.pl)
% Bardzo wczesnym pierwowzorem szablonu był
% The Legrand Orange Book
% Version 2.1 (26/09/2018)
%
% Modifications to LOB assigned by %JK
%%%%%%%%%%%%%%%%%%%%%%%%%%%%%%%%%%%%%%%%%


\chapter*{Wstęp}

\par W branży programistycznej istnieje bardzo dużo języków programowania oraz różnych środowisk uruchomieniowych dla nich przeznaczonych. Podczas tworzenia oprogramowania niezbędny jest dla programisty kompilator. Zamienia on kod programu napisanego w konkretnym języku programowania, na zrozumiały dla środowiska uruchomieniowego ciąg instrukcji. Warto podkreślić, że języki programowania, które są proste do zrozumienia i opanowania dla programistów oraz pozwalają na pisanie programów, które można uruchomić na różnych urządzeniach i umożliwiają tworzenie bardzo zróżnicowanych rodzajów oprogramowania, są o wiele częściej używane niż inne. Powoduje to, że powstaje dużo różnych bibliotek i frameworków ułatwiających i automatyzujących tworzenie oprogramowania.

\par Jednym z popularnych języków wśród programistów jest język JavaScript, który może być uruchomiany w przeglądarkach internetowych lub w maszynie wirtualnej takiej jak Node.js. Innym z popularnych środowisk uruchomieniowych jest platforma .NET, dla której istnieje wiele kompilatorów różnych języków. Wykorzystanie istniejących bibliotek, frameworków czy modułów napisanych w języku JavaScript w niezmienionej formie na platformie .NET, umożliwi programistom na tworzenie bardziej uniwersalnego kodu.

\par \textbf{Celem} niniejszej pracy jest \textbf{zaprojektowanie} oraz \textbf{implementacja kompilatora} języka \textbf{JavaScript} na kod \textbf{IL Assembler} uruchamiany na \textbf{platformie .NET}. Następnie w celu sprawdzenia poprawności działania kompilatora, zostaną przeprowadzone testy przy pomocy prostych implementacji kodu JavaScript. Zostaną również zaprojektowane oraz zaimplementowane dwa bardziej skomplikowane testy, do \textbf{porównania maszyn wirtualnych Node.js oraz .NET}. Zostanie także porównany kod assemblera generowanego przez kompilator .Net Framework języka C\# z kodem generowanym przez implementowany kompilator.

\par Praca podzielona jest na cztery części. Pierwsza część opisuje pojęcia i technologie wykorzystywane do realizacji tego projektu oraz technologie pokrewne, które w pewnym stopniu realizują cel pracy lub realizują podobne założenia. Druga część poświęcona jest zaplanowaniu zakresu implementacji tworzonego kompilatora oraz analizie języka JavaScript. Kolejna część opisuje sposób implementacji tego kompilatora na podstawie zdefiniowanych założeń. Ostatnia część opisuje przeprowadzone testy realizowane w ramach pracy oraz przedstawia wyniki ich działania.
