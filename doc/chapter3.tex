\chapter{Implementacja aplikacji kompilatora}
\label{rozdzial3}

\section{Parser}
\par W projekcie zostanie wykorzystane narzędzie ANTLR, które można z następującego linku \url{https://www.antlr.org/download.html} (dostępne: 28.05.2021). Należy wydzielić osobny folder w którym będą znajdować się pliki z definicjami leksemów oraz gramatyki. Do danego folderu należy przenieść plik wykonywalny (\textit{antlr.jar}) narzędzia ANTLR. Aby uruchomić narzędzie wymagane jest posiadanie środowiska wykonawczego Java.
\par W celu wygenerowania plików parsera dla środowiska .NET należy wywołać komendę przedstawioną w algorytmie \ref{alg:antlr}. W folderze w którym zostanie wykonana komenda znajdują się aktualnie dwa pliki: 
\begin{itemize}
  \item \textit{JavaScriptLexer.g4} - plik zawierający definicję leksemów
  \item \textit{JavaScriptParser.g4} - plik zawierający definicję gramatyki
\end{itemize}
Po wykonaniu komendy generowania, powstaną cztery pliki wynikowe: 
\begin{itemize}
  \item \textit{JavaScriptLexer.cs}
  \item \textit{JavaScriptParser.cs}
  \item \textit{JavaScriptParserBaseListener.cs}
  \item \textit{JavaScriptParserListener.cs}
\end{itemize}
Są to pliki kodu źródłowego parsera dla zdefiniowanych wcześniej leksemów oraz gramatyki. W celu użycia parsera, należy utworzyć własną klasę, która będzie dziedziczyć po klasie \textit{JavaScriptParserBaseListener} zawierająca definicje nagłówków funkcji.

\begin{lstlisting}[caption=Komenda uruchamiająca narzędzie ANTLR, label=alg:antlr]
  java -jar antlr.jar -Dlanguage=CSharp *.g4
\end{lstlisting}

\subsection{Analiza leksykalna}
\par W celu przeanalizowania pliku wejściowego kodu JavaScript pod kątem leksykalnym i utworzeniu tokenów, należy zdefiniować odpowiednie reguły w pliku \textit{JavaScriptLexer.g4}. 
\par Na samym początku pliku musi się znajdować deklaracja, że ten plik zawiera reguły do analizy leksykalnej. 

\begin{lstlisting}[caption=Deklaracja pliku do analizy leksykalnej, label=alg:antlrLex1]
  lexer grammar JavaScriptLexer;
\end{lstlisting}

Następnie w pliku powinny się znajdować reguły tokenów. Definiuje się je, nadając nazwę tokenu, a po dwukropku wpisuje się jedno lub wiele wyrażeń dopasowania oddzielonymi znakami `|'. Nazwy tokenu muszą zaczynać się wielką literą.


\begin{lstlisting}[caption=Przykład definicji tokenów dla znaków operacji porównania, label=alg:antlrLex2]
  LESS_THAN:           '<';
  MORE_THAN:           '>';
  LESS_THAN_EQUALS:    '<=';
  GREATER_THAN_EQUALS: '>=';
  EQUALS:              '==';
  NOT_EQUALS:          '!=';
  IDENTITY_EQUALS:     '===';
  IDENTITY_NOT_EQUALS: '!==';
\end{lstlisting}

\begin{lstlisting}[caption=Przykład definicji tokenów dla znaków słów kluczowych, label=alg:antlrLex3]
  CONSOLE_LOG: 'console.log';
  VAR:         'var';
  CONTINUE:    'continue';
  FOR:         'for';
  WHILE:       'while';
  FUNCTION:    'function';
  LENGTH:      'length';
  RETURN:      'return';
\end{lstlisting}

\par Można również zdefiniować regułę pomocniczą, która nie będzie interpretowana jako token, ale jej dopasowanie będzie uwzględniane, przy wykorzystaniu jej w innej regule. Aby stworzyć taką regułę pomocniczą, należy użyć instrukcji \texttt{fragment} przed nazwą tokenu.

\begin{lstlisting}[caption=Przykład definicji tokenu dla wielu dopasowań, label=alg:antlrLex4]
  NUMBER
  : INTEGER_NUMBER '.' [0-9] [0-9_]*
  | '.' [0-9] [0-9_]*
  | INTEGER_NUMBER
  ;

fragment INTEGER_NUMBER
  : '0'
  | '-'?[1-9] [0-9_]*
  ;
\end{lstlisting}

% https://sodocumentation.net/antlr/topic/3271/lexer-rules-in-v4
\par Można również zdefiniować regułę, która zamiast tworzyć token, będzie dane dopasowanie pomijać. Aby tego dokonać, trzeba dołączyć do danej reguły, wywołanie komendy leksera \texttt{skip}. Istnieją również inne komendy takie jak: \texttt{channel(n)} - służący do zmiany kanału emitowanych tokenów; \texttt{type(n)} - zmieniający typ tokenu; \texttt{mode(n)}, \texttt{pushMode(n)}, \texttt{popMode}, \texttt{more} - kontrolujące tryb działania leksera. \footcite{AntlrLexRules}

\begin{lstlisting}[caption=Przykład użycia komendy \texttt{skip}, label=alg:antlrLex5]
  MULTI_LINE_COMMENT:  '/*' .*? '*/'   -> skip;
\end{lstlisting}

\subsection{Gramatyka}
Po zdefiniowaniu tokenów, można rozpocząć definicję gramatyki. Tak jak było w przypadku pliku z leksemami, należy w pliku \textit{JavaScriptParser.g4} zadeklarować, że plik zawiera reguły gramatyki oraz również wskazać plik zawierający tokeny.

\begin{lstlisting}[caption=Deklaracja pliku do analizy gramatyki, label=alg:antlrGr1]
  parser grammar JavaScriptParser;

  options {
    tokenVocab=JavaScriptLexer;
  }
\end{lstlisting}

Definicje gramatyk deklaruje się w identyczny sposób jak tworzyło tokeny. Najpierw należy podać nazwę reguły a pod dwukropku, podajemy jedna lub wiele definicji oddzielonymi znakiem `|'. Zdefiniowana gramatyka może być wykorzystana w innych definicjach. 

\begin{lstlisting}[caption=Przykład definicji reguły gramatyki, label=alg:antlrGr1]
  booleanValue
  : BOOLEAN
  ;

  constantValue
  : booleanValue
  | stringValue
  | numberValue
  ;
\end{lstlisting}

\par Dla każdej zdefiniowanej reguły, przy pomocy narzędzia ANTLR, zostaną utworzone dwie funkcję w wygenerowanym pliku parsera. Pierwsza będzie się wykonywać przed, a druga po analizie reguł. Nazwy wygenerowanych funkcji będą identyczne jak nazwa reguły z dopisanym przedrostkiem \textit{Enter} dla funkcji wywoływanej przed analizą reguły oraz \textit{Exit} dla funkcji po analizie. Jako argument funkcji jest przekazywany kontekst parsera, dzięki któremu można pobrać wartość leksemu, lub też sprawdzić która definicja jest analizowana.

\begin{lstlisting}[language=CSharp, caption=Przykład wygenerowanych funkcji gramatyki parsera, label=alg:antlrGr2]
  /// <summary>
  /// Enter a parse tree produced by <see cref="JavaScriptParser.constantValue"/>.
  /// <para>The default implementation does nothing.</para>
  /// </summary>
  /// <param name="context">The parse tree.</param>
  public virtual void EnterConstantValue([NotNull] JavaScriptParser.ConstantValueContext context) { }
  /// <summary>
  /// Exit a parse tree produced by <see cref="JavaScriptParser.constantValue"/>.
  /// <para>The default implementation does nothing.</para>
  /// </summary>
  /// <param name="context">The parse tree.</param>
  public virtual void ExitConstantValue([NotNull] JavaScriptParser.ConstantValueContext context) { }
\end{lstlisting}

\par Domyślna implementacja wygenerowanych metod, nie posiadają żadnych instrukcji, dlatego też nie trzeba pisać własnych implementacji dla każdej z reguł. W przypadku kiedy chcemy napisać implementacje tych metod, jak było wcześniej wspominane, trzeba stworzyć własną klasę dziedziczącą po wygenerowanej klasie i podać jej instancję, jako klasa nasłuchująca (\textit{listner}) dla parsera.

\section{Kompilator}

\par Program kompilatora wymaga podania jednego argumentu, który jest ścieżką do pliku źródłowego, zawierającego kod JavaScript. Jego wynikiem jest plik o tej samej nazwie z rozszerzeniem ``\textit{.il}'' zawierający kod assemblera platformy .NET. Zawartość pliku źródłowego jest analizowana przez parser, który wywołuje odpowiednie metody przy dopasowaniu reguł gramatyki.
\par W funkcji głównej \textit{Main} kompilatora znajduje się: otwarcie pliku źródłowego, utworzenie instancji modułu do generowania pliku wyjściowego, konfiguracja wcześniej wygenerowanego parsera oraz jego uruchomienie. Po zakończeniu analizy pliku źródłowego, plik jest zamykany, a następnie tworzony jest plik wyjściowy.

\begin{lstlisting}[language=CSharp, caption=Funkcja \textit{Main} kompilatora, label=alg:komMain]
  static void Main(string[] args) {
    if (args.Length < 1) {
      throw new ArgumentException("Needs file name as argument");
    }

    using (AsmGenerator.Create(args[0])) {
      using (FileStream fs = File.OpenRead(args[0])) {
        AntlrInputStream imputStream = new AntlrInputStream(fs);
        JavaScriptLexer lexer = new JavaScriptLexer(imputStream);
        CommonTokenStream commonTokenStream = new CommonTokenStream(lexer);
        JavaScriptParser parser = new JavaScriptParser(commonTokenStream);
        parser.BuildParseTree = true;
        JavaScriptParser.ParseContext context = parser.parse();

        JavaScriptListner listner = new JavaScriptListner();
        ParseTreeWalker walker = new ParseTreeWalker();
        walker.Walk(listner, context);
      }
    }
    Console.WriteLine("Done.");
  }
\end{lstlisting}

\subsection{Wywołania parsera}

Do przechwytywania reguł gramatyki została utworzona klasa o nazwie \textit{JavaScriptListner} dziedzicząca po wygenerowanej klasie \textit{JavaScriptParserBaseListener}. Zawiera ona nadpisane metody parsera w których wywoływane są odpowiednie funkcje z modułów obsługujących funkcjonalności. Funkcjonalności zostały podzielone na moduły i przeniesione do osobnych klas, ze względu na czytelność kodu.
\par Niektóre z funkcji wywołują jedynie funkcjonalność modułu, a inne sprawdzają dodatkowe warunki na podstawie dostarczonego kontekstu. Są przypadki gdzie, trzeba sprawdzić która z definicji reguł jest analizowana i w zależności od tego wywołać odpowiednią metodę z konkretnego moduł.

\begin{lstlisting}[language=CSharp, caption=Przykład prostej funkcji listnera, label=alg:komListner1]
  public override void ExitIfStatement(JavaScriptParser.IfStatementContext context)
  {
    conditionModule.EndIfStatement();
  }
\end{lstlisting}

\par Klasa \textit{JavaScriptListner} ma również za zadanie przy pomocy kontekstu, pobierać nazwy oraz wartości znajdujące się w pliku źródłowym dla konkretnych reguł. Przykładowo musi pobrać nazwę deklarowanej zmiennej lub też przypisywanej do niej wartości.

\begin{lstlisting}[language=CSharp, caption=Przykład funkcji listnera z sprawdzeniem kontekstu, label=alg:komListner2]
  public override void ExitIdentifierValue(JavaScriptParser.IdentifierValueContext context) {
    string value = context.GetChild(0).GetText();
    if(context.ChildCount > 1) {
      arrayModule.CreateTableVariable(value);
    } else {
      variableModule.CreateVariable(value);
    }
  }
\end{lstlisting}

\subsection{Zarządzanie stosem zmiennych}

\par W celu zarządzania oraz przechowywania informacji o wartościach kodu źródłowego, została utworzona statyczna klasa \texttt{Store}. Zawiera ona następujące elementy:
\begin{itemize}
  \item \texttt{Stack} - stos główny kompilatora
  \item \texttt{LabelStack} - stos indexów dla nazw instrukcji warunkowych oraz pętli
  \item \texttt{FunctionCallArgCounterStack} - stos służący do zliczania argumentów funkcji
  \item \texttt{\_labelStackCounter} - licznik indexów etykiet
  \item \texttt{Variables} - słownik zdefiniowanych zmiennych
  \item \texttt{Functions} - słownik informacji definiowanych funkcji
  \item \texttt{ProcessingFunction} - nazwa aktualnie przetwarzanej funkcji.
\end{itemize}
\par Elementem który przechowuje główny stos oraz słownik zmiennych jest klasa \texttt{StoreItem}. Zawiera on niezbędne informacje o przetwarzanym elemencie. Dzięki tej klasie można przechować następujące informacje:
\begin{itemize}
  \item \texttt{ItemType} - definiuje czy dany element jest zmienną o danym typie, zmienną tablicową, elementem tablicy, argumentem funkcji czy znakiem arytmetycznym lub porównania.
  \item \texttt{IsVariable} - definiuje czy element jest zmienną
  \item \texttt{IsTemporary} - definiuje czy element jest zmienną tymczasową
  \item \texttt{IsInitialized} - definiuje czy zmienna jest dopisana do listy do zainicjalizowania
  \item \texttt{IsFunctionParam} - definiuje czy zmienna jest parametrem funkcji
  \item \texttt{ParamPosition} - definiuje pozycję w liście argumentów funkcji
  \item \texttt{Parent} - element z którego powstał dany element
  \item \texttt{TableIndex} - index zmiennej tablicowej
\end{itemize}

\par Dla słownika informacji dla zdefiniowanych funkcji została stworzona klasa \texttt{FunctionStore}. Przechowuje ona:
\begin{itemize}
  \item \texttt{Variables} - słownik zmiennych lokalnych funkcji
  \item \texttt{Params} - słownik parametrów funkcji
  \item \texttt{Name} - nazwę funkcji
  \item \texttt{IsUsed} - informacje, czy dana funkcja została wywołana co najmniej raz
  \item \texttt{ReturnValue} - wartość zwracana przez funkcję
\end{itemize}

\subsection{Moduły}

Każdy z modułów wykorzystuje \texttt{Store} do przetwarzania instrukcji kodu źródłowego i wywołuje odpowiednie funkcje klasy generującej kod assemblera. Moduły zostały podzielone ze względu na grupy implementowanych funkcjonalności.

\subsubsection{Obsługa standardowego wyjścia}

W JavaScript funkcją wyświetlającą tekst na standardowym wyjściu (konsoli) służy funkcja \texttt{log()} z standardowej klasy \texttt{console}. Jako że założony zakres implementacji kompilatora nie uwzględnia implementacji klas, wywołanie tej funkcji zostało zdefiniowane jako słowo kluczowe. Dzięki temu zabiegowi można zasymulować wywołanie tej metody, jakby rzeczywiście została wywoływana ze standardowej klasy.

\begin{lstlisting}[caption=Definicja gramatyki dla funkcji standardowego wyjścia, label=alg:consolelogGram]
  writeStdOutput
  : CONSOLE_LOG OPEN_PAREN constantValue CLOSE_PAREN
  | CONSOLE_LOG OPEN_PAREN identifierValue CLOSE_PAREN
  | CONSOLE_LOG OPEN_PAREN arithmeticOperation CLOSE_PAREN
  | CONSOLE_LOG OPEN_PAREN functionCall CLOSE_PAREN
  ;
\end{lstlisting}

\par Zostały zdefiniowane cztery gramatyki dla różnych rodzajów parametrów: wartości stałej (takich jak liczby, czy ciągi znaków), zmiennej, operacji arytmetycznej oraz wywołania funkcji.

\par Po realizacji definicji gramatyki parametru, zostanie zawsze odłożona na główny stos wartość którą chcemy wyświetlić na konsoli. W tym celu używana jest funkcja o nazwie \texttt{WriteStdOutput} z modułu \texttt{ConsoleModule}. Ściąga ona ze stosu wartość, a następnie w zależności od tego, czy jest to zmienna tablicowa wywołuje odpowiednią funkcję generatora assemblera.

\begin{lstlisting}[language=CSharp, caption=Definicja gramatyki dla funkcji standardowego wyjścia, label=alg:consolelogMdoule]
  public void WriteStdOutput(){
    StoreItem item = Store.PopStack();
    if(item.IsType(StoreItemType.ARRAY)){
      asmGenerator.WriteArrayToStdOutput(item);
    }else {
      asmGenerator.WriteToStdOutput(item);
    }
    asmGenerator.EmptyLine();
  }
\end{lstlisting}

\par Wyświetlenie pojedynczej wartości wymaga jedynie wygenerowania kodu ładującego daną wartość na stos oraz wywołanie odpowiedniej funkcji do wyświetlenia. W przypadku tablic został zastosowany zabieg, polegający na połączenie wszystkich elementów w ciąg znaków i wyświetlenie otrzymanej wartości na ekranie.

\begin{lstlisting}[language=IL, caption=Przykład wygenerowanego kodu assemblera dla funkcji wyjścia standardowego, label=alg:consolelogAsm]
  // Wyświetlenie pojedynczej wartości
  ldstr "tekst 123"
  call void [mscorlib]System.Console::WriteLine(string)

  // Wyświetlenie tablicy
  ldstr   "["
  ldstr   ", "
  ldloc v_tab
  call string [mscorlib]System.String::Join<int32>(string, class [mscorlib]System.Collections.Generic.IEnumerable`1 <!!0/*int32*/>)
  ldstr   "]"
  call string [mscorlib]System.String::Concat(string, string, string)
  call void [mscorlib]System.Console::WriteLine(string)
\end{lstlisting}

\subsubsection{Obsługa zmiennych}

\par Moduł obsługi zmiennych obejmuje funkcjonalności związanych z definicją elementów stałych, zmiennych, obsługi deklaracji zmiennych oraz obsługi rzutowania typów zmiennych. \par Przy analizowaniu reguły gramatyki identyfikującej liczbę, ciąg znaków, wartość liczbową, czy nazwę identyfikatora, wywoływane są dla nich odpowiednie metody które utworzą element opisujący dane wartości i dopiszą je na szczyt głównego stosu. Deklaracja zmiennych przebiega w ten sam sposób, ponieważ przypisanie do listy zmiennych następuje przy pierwszym wystąpieniu przypisania wartości do danej zmiennej. Przy tych operacjach nie jest tworzony kod asemblerowy.

\begin{lstlisting}[language=CSharp, caption=Implementacja funkcji tworzącej element opisujący stałą liczbową, label=alg:variable1]
  public void CreateNumber(string value) {
    StoreItem item;
    if (value.Contains(".")) {
      item = StoreItem.CreateDouble(value);
    } else {
      item = StoreItem.CreateInteger(value);
    }
    Store.PushStack(item);
  }
\end{lstlisting}

\par W tym module znajduje się jeszcze funkcjonalność rzutowania typów zmiennych. Jest ona potrzebna, ponieważ język JavaScript jest językiem z dynamicznym typowaniem zmiennych i pozwala na operacje na różnych typach zmiennych. Funkcjonalność jest wykorzystywana przykładowo przy operacjach arytmetycznych, kiedy chcemy dodać do siebie wartość całkowitą do wartości rzeczywistej. W takim przypadku następuje rzutowanie zmiennej całkowitej na zmienną rzeczywistą.
\par Funkcja rzutowania przyjmuje jako argument element opisujący zmienną oraz typ docelowy. Na podstawie dostarczonego elementu tworzona jest jego kopia, następnie nowa zmienna jest inicjalizowana poprzez dopisanie jej do listy zadeklarowanych zmiennych. Na końcu wywoływana jest funkcja generująca kod assemblera, służąca do rzutowania wartości.

\begin{lstlisting}[language=CSharp, caption=Implementacja funkcji rzutującej zmienne, label=alg:variable2]
  public StoreItem CastVariable(StoreItem item, StoreItemType itemType){
    if(item.ItemType == itemType){
      return item;
    }

    StoreItem vessel = item.CreateCastVariable(itemType);
    asmGenerator.InitializeVariable(vessel);
    asmGenerator.CastVariable(item, vessel);

    asmGenerator.Comment($"{item.Print} -> {vessel.Print}\n");
    return vessel;
  }
\end{lstlisting}

\par Funkcja generująca asemblera do rzutowania zmiennych, początkowo sprawdza, czy dany element jest stałą czy zmienną. W przypadku kiedy podany zostanie element opisujący wartość stałą, generator najpierw utworzy zmienną, a następnie przypisze do niej wartość stałej. Jest to konieczne, ponieważ rzutowanie typów zmiennych nie może odbywać się na wartościach stałych. Następnie w zależności od tego czy będzie odbywać się rzutowanie na wartość rzeczywistą czy łańcuch znaków, zostanie wygenerowany odpowiedni kod.

\begin{lstlisting}[language=IL, caption=Kod assemblera rzutowania wrtości całkowitej na wartość rzeczywistą, label=alg:variable3]
  // załadowanie stałej wartości całkowitej "5"
  ldc.i4 5
  // przypisanie wartości "5" do zmiennej "v_tmp_36"
  stloc v_tmp_36

  // załadowanie wrtości "v_tmp_36"
  ldloc v_tmp_36
  // konwersja wartości do typu rzeczywistego
  conv.r4
  // zapisanie wyniku konwersji do zmiennej "v_tmp_35"
  stloc v_tmp_35
\end{lstlisting}

\par W przypadku rzutowania zmiennych na łańcuch znaków, należy wywołać procedurę konwertującą wartość \texttt{ToString()}. Przy wywołaniu trzeba podać typ źródłowy wartości jaką chcemy rzutować. Są to odpowiednio:
\begin{itemize}
  \item \texttt{Int32} dla typu całkowitego
  \item \texttt{Single} dla typu rzeczywistego
  \item \texttt{Boolean} dla typu logicznego
\end{itemize}

\begin{lstlisting}[language=IL, caption=Kod assemblera rzutowania wrtości rzeczywistej na wartość łańcucha znaków, label=alg:variable4]
  // załadowanie stałej wartości całkowitej "4.7"
  ldc.r4 4.7
  // przypisanie wartości "5" do zmiennej "v_tmp_34"
  stloc v_tmp_34
  // załadowanie adresu "v_tmp_34"
  ldloca.s v_tmp_34
  // wywołanie procedury konwersji typu rzeczywistego na wartość łańcucha znaków
  call instance string [mscorlib]System.Single::ToString()
  // zapisanie wyniku konwersji do zmiennej "v_tmp_33"
  stloc v_tmp_33
\end{lstlisting}

\par Przy generowaniu kodu assemblera dla tej konwersji, w przypadku kiedy znajduje się ona w ciele funkcji, są wstawiane znaczniki. Będą one zamieniane na instrukcje asemblerowe dopiero po przeanalizowaniu całego kodu pliku wejściowego. Taki zabieg wynika z konieczności, podania typu wartości źródłowej przy wywołaniu funkcji \texttt{ToString()}, a przy analizowaniu funkcji dany typ może być nie znany.

\begin{lstlisting}[language=IL, caption={Przykład kodu assemblera rzutowania wrtości w funkcji, przed podmianą znacznika i po podmianie}, label=alg:variable4]
  // Znacznik opisuje, że w funkcji o nazwie "convert" należy obsłużyć rzutowanie zmiennej o nazwie "a"
  ldarga v_a
  #CAST_VARIABLE#a@convert
  stloc v_tmp_3

  // Kod asemblera po zamianie znacznika
  ldarga v_a
  call instance string [mscorlib]System.Single::ToString()
  stloc v_tmp_3
\end{lstlisting}

\subsubsection{Obsługa działań arytmetycznych}

\par Moduł obsługi działań arytmetycznych zawiera definicje elementu znaku działań matematycznych, obsługę funkcjonalności działań matematycznych, operacji przypisania oraz inkrementacji zmiennej.
\par Definiowanie elementu znaku działań matematycznych odbywa się w identyczny sposób jak przy definicji wartości stałych i zmiennych. Na główny stos aplikacji zostaje wrzucony element opisujący działanie, a raczej znak działania matematycznego.
\par Przy definiowaniu gramatyki dla działań matematycznych należy zdefiniować osobne reguły dla działań dodawania i odejmowania oraz mnożenia i dzielenia, w celu zachowania zasady kolejności wykonywania działań.

\begin{lstlisting}[caption=Definicja gramatyki dla funkcji działań matematycznych, label=alg:arytm1]
  arithmeticOperation
  : arithmeticOperation arithmeticAdditiveSign arithmeticOperationHigher
  | arithmeticOperationHigher
  ;

  arithmeticOperationHigher
  : arithmeticOperationHigher arithmeticMultiplpicativeSign value
  | value
  ;

  arithmeticAdditiveSign
  : PLUS
  | MINUS
  ;

  arithmeticMultiplpicativeSign
  : MULTIPLY
  | DIVIDE
  ;
\end{lstlisting}


\par Przy wywołaniu funkcji przetworzenia działania matematycznego, na stosie głównym znajdować się będą trzy elementy uczestniczące w działaniu. Będzie to kolejno dwa elementy opisujące wartości oraz element opisujący znak operacji. Następnie weryfikowane jest jakiego typu są elementy. W przypadku różnych typów danych wykonywane są funkcje rzutowania wartości na wspólne typy danych.

\begin{lstlisting}[language=CSharp, caption=Implementacja funkcji obsługującej działania matematyczne cz.1, label=alg:arytm2]
  public void ProcessArithmeticOperation() {
    StoreItem arg2 = Store.PopStack();
    StoreItem sign = Store.PopStack();
    StoreItem arg1 = Store.PopStack();

    bool stringOperation = false;

    if (StoreItem.IsAnyType(StoreItemType.STRING, arg1, arg2)) {
      if (sign.Value != "+") {
        throw new InvalidOperationException($"Operation is not allowed for strings.");
      }
      stringOperation = true;

      arg1 = variableModule.CastVariable(arg1, StoreItemType.STRING);
      arg2 = variableModule.CastVariable(arg2, StoreItemType.STRING);
    }
    if(sign.Value == "/" || StoreItem.IsAnyType(StoreItemType.DOUBLE, arg1, arg2)){
      arg1 = variableModule.CastVariable(arg1, StoreItemType.DOUBLE);
      arg2 = variableModule.CastVariable(arg2, StoreItemType.DOUBLE);
    }

    ...
\end{lstlisting}

\par Następnie wywoływane są funkcje generujące kod asemblera. W pierwszym kroku obie wartości są ładowane na stos, a następnie w zależności od tego, czy operacja jest wykonywana na typach łańcuchów znaków, czy na wartościach liczbowych, wykonywane są funkcje do łączenia wartości typu łańcuchowego lub też do wykonania operacji na liczbach.

\begin{lstlisting}[language=CSharp, caption=Implementacja funkcji obsługującej działania matematyczne cz.2, label=alg:arytm3]
  ...

  asmGenerator.Load(arg1);
  asmGenerator.Load(arg2);
  if (stringOperation) {
    asmGenerator.ConcatStrings();
  } else {
    asmGenerator.ExecuteArithmeticOperation(sign);
  }

  ...
\end{lstlisting}

\par Na końcu tworzona jest nowa zmienna do której przypisywana będzie wartość wyniku powyższej operacji oraz wynik zostanie wrzucony na główny stos kompilatora. W przypadku kiedy jedna z wartości była już elementem tymczasowym, zamiast tworzenia nowego elementu tymczasowego, wynik zostanie przypisany już do istniejącego.

\begin{lstlisting}[language=CSharp, caption=Implementacja funkcji obsługującej działania matematyczne cz.3, label=alg:arytm4]
  ...

  StoreItem resultItem;
  if (arg1.IsTemporary) {
    resultItem = arg1;
  } else if (arg2.IsTemporary) {
    resultItem = arg2;
  } else {
    resultItem = StoreItem.CreateTemporaryVariable(arg1.ItemType);
    resultItem.Parent = arg1;
    asmGenerator.InitializeVariable(resultItem);
  }

  Store.PushStack(resultItem);
  asmGenerator.StoreVariable(resultItem);
\end{lstlisting}

\par ... Opisać asemblera do powyższego oraz resztę metod

\subsubsection{Obsługa wyrażeń warunkowych}
\subsubsection{Obsługa pętli}
\subsubsection{Obsługa tablic}
\subsubsection{Obsługa funkcji}

\subsection{Generator asemblera}
